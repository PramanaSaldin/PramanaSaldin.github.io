\documentclass{scrartcl}
\usepackage{bookmark}
\usepackage[utf8]{inputenc}
\usepackage{pramanahonors}
\hypersetup{pdfencoding=unicode}
\newcommand{\qedwhite}{\hfill \ensuremath{\Box}}


\title{G\r{a}rding's Inequality}
\author{Pramana Saldin}
\date{Fall 2022}

\begin{document}
\newgeometry{margin=1in}

\maketitle

\begin{abstract}
    \begin{center}
        \textbf{Abstract}
    \end{center}
    This is an expository paper on the prerequisites for 
    G\r{a}rding's inequality. Proposed in \cite{original-garding} by Lars G\r{a}rding, this inequality has applications in the study of weak solutions to elliptic partial differential equations. 
    
    This will begin by introducing Lebesgue integration, a stronger form 
    of Riemann integration, and continue into function spaces, specifically $L_p$ and Sobolev spaces. Then we will discuss differential operators. Finally, we will state G\r{a}rding's inequality and give one application. 
    Discussion of applications is adapted from 
    \cite{renardy2004introduction}.
\end{abstract}
\tableofcontents
\restoregeometry

\section{Lebesgue integration}

\subsection{A motivating example}

The typical Riemann integration we use works for many functions, but
not for all. Consider the \textit{Dirichlet function} 
\begin{equation}
    \label{eq:dirichleteq}
    1_\QQ = \begin{cases}
        1 &\text{if } x\in \QQ \\
        0 &\text{if } x\notin \QQ,
    \end{cases}
\end{equation}
on the interval $[0,1]$.
Since rationals and irrationals are dense in 
the reals, all upper Darboux sums will come out to $1$, and 
all lower Darboux sums will come out to $0$. We conclude that 
the function is not Riemann integrable. The \textit{Lebesgue integral} 
seeks to give a larger space of integrable functions, 
including Equation \ref{eq:dirichleteq}.

\subsection{The Lebesgue outer measure}

This definition comes from Prof. Rycroft and 
\cite{nlab:lebesgue_measure}.

\begin{definition}
    For an open interval $I=(a,b)$, let $\pa{I} = b-a$.
    The \vocab{Lebesgue outer measure} for a subset $E$ of the real 
    numbers, denoted $\lambda(E)$, is defined as 
    \[
        \lambda(E) := \inf\ps{\sum_{j=1}^{\infty}\pa{I_j}: (I_j)_{j\in \NN} \text{ such that } B \subseteq \bigcup_{i=1}^\infty I_j}.
    \]
\end{definition}
Sets used in the Lebesgue integral must satisfy the \vocab{Cath\'edory criterion}:
\[
    \lambda(B) = \lambda(A\cap B) + \lambda(A\cap B^C), \quad \text{for all }A \subseteq\RR.
\]

\subsection{The Lebesgue integral}

While Riemann integrals start with splitting the \textit{domain} into smaller and smaller intervals, the Lebesgue integral 
splits the \textit{range} into smaller and smaller intervals.

Formally, we define \textit{indicator functions} as 
\[
    1_S(x) = \begin{cases}
        1 &\text{if }x\in S \\
        0 &\text{if }x\notin S.
    \end{cases}
\]
Then we can create \textit{simple function} in terms of sums of 
these indicator functions and for subsets $S_i \subseteq \RR$,
\[
    s(x) = \sum_{n}a_n1_{S_n}.
\]
For this simple function, we can define an integral as 
\[
    \int s d\lambda = \sum_{n}\lambda(S_n)a_n.
\]
\begin{definition}
    Given that a function is non-negative, we then define the 
    \vocab{Lebesgue integral} as 
    \[
        \int fd\lambda := \sup\ps{\int sd\lambda:0\leq s\leq f, s \text{ is a simple function}}.
    \]
    If we want to deal with functions with positive and negative ranges, 
    then we can just apply the Lebesgue integral to each part separately.
\end{definition}

\begin{example}
    For the Dirichlet function,
    \begin{align*}
        \int_{0}^{1}1_\QQ(x) d\lambda(x) &= \lambda([0,1]\cap \QQ)\cdot1  + \lambda([0,1]\setminus \QQ)\cdot 0 \\
        &= 0\cdot 1 + 1\cdot 0 \\
        &= 0.
    \end{align*}
    This matches with out intuition of the ``amount'' of rationals 
    compared to irrationals on the interval $[0,1]$.
    \qedwhite
\end{example}

\begin{theorem}
    Functions that are Riemann-integrable on a 
    closed interval are Lebesgue-integrable.
\end{theorem}

\section{The Lp space of functions}

\subsection{Normed vector spaces}

\begin{definition}
    A \vocab{norm} is a way of formalizing ``length'' in certain spaces. It 
    is denoted $\pn{\cdot}$. 
    There are 4 properties a norm must satisfy:
    \begin{enumerate}
        \item $\pn{x} \geq 0$.
        \item $\pn{x}=0\implies x=0$.
        \item For a scalar $\alpha$, $\pn{\alpha x} = \pa{\alpha}\pn{x}$.
        \item The \vocab{triangle inequality} holds: $\pn{x+y} \leq \pn{x} + \pn{y}$.
    \end{enumerate}
\end{definition}
A vector space equipped with a norm is a \vocab{normed vector space}. 
We also have an induced distance function by the norm, 
$d(x,y) = \pn{y-x}$. Therefore, a normed space automatically a metric space.

\subsection{The Lp space and norm}

\begin{definition}
    The \vocab{Lebesgue space ($L^p$ space)} is the space of functions 
    from a set that can be assigned a Lebesgue measure $\Omega \subseteq\RR^n$ to $\RR$ or $\CC$ 
    such that the following norm converges:
    \[
        \pn{f}_{L^p} := \pp{\int_\Omega\pa{f}^pd\mu}^{\frac{1}{p}}
    \]
    We call this the \vocab{$L^p$ norm} of $f$. 
    The $L^p$ space of functions $f:\Omega\to \RR$ is denoted $L^p(\Omega)$.
\end{definition}

\sidenote{
\begin{remark}
    Most of the norm properties are immediately satisfied, except for the triangle inequality, $\pn{x+y} \leq \pn{x} + \pn{y}$, which follows from \textit{Minkowski's inequality}.
\end{remark}}

\begin{proposition}
    The Lebesgue space is a vector space, with
    \[
        (f+g)(x) = f(x) + g(x) \in L^p
    \]
    \[
        (\lambda f)(x) = \lambda f(x) \in L^p
    \]
\end{proposition}

\subsection{Bilinear forms and inner products}

\begin{definition}
    A \vocab{bilinear form} on a vector space $V$ is a function $B[\cdot,\cdot]:V\times V \to K$ \sidenote{$K$ is the field which $V$ is a vector space over. Usually $K$ is $\RR$ or $\CC$.}
    such that $\pv{x,y}$ is linear in $x$ and $y$. In symbols, 
    \[
        B[ax_1 + bx_2, y] = a\cdot B[x_1,y] + b\cdot B[x_2,y], 
    \]
    \[
        B[x, ay_1+by_2] = a\cdot B[x,y_1] + b\cdot B[x,y_2].
    \]
\end{definition}

\begin{definition}
    An \vocab{inner product} is an bilinear form $\pv{\cdot, \cdot}$ such 
    that 
    \begin{enumerate}
        \item $\pv{ax_1 + bx_2, y} = a\pv{x_1,y} + b\pv{x_2,y}$, $\pv{x, ay_1+by_2} = a\pv{x,y_1} + b\pv{x,y_2}$.
        \sidenote{
        \begin{remark}
            The complex conjugate is not useful in $\RR$, but if the function is mapping to $\CC$, it is commonly called a \textit{Hermitian inner product}, and it makes use of the complex conjugate.
        \end{remark}
        }
        
        \item \textit{Reflexive:} $\pv{y,x} = \ol{\pv{x,y}}$, where $\ol{z}$ denotes the \textit{complex conjugate} of $z$.

        \item \textit{Positive Definite:} $\pv{x,x} = 0$ if $x = 0$, and $\pv{x,x}>0$ if $x\neq 0$.
    \end{enumerate}
    A vector space equipped with an inner product is an \vocab{inner product space}.
\end{definition}

\begin{example}
    $L^2$ has an induced inner product with respect to a measure $\mu$:
    \[
        \pv{f,g}_{L^2} = \int_\Omega f gd\mu.
    \]
    In fact, $L^2$ is the only $L^p$ space with this property.
    \sidenote{
    \begin{remark}The inner product induces a norm, which therefore induces a metric.
    \[
        \pn{x} = \sqrt{\pv{x,x}}.
    \]
    \[
        d(x,y) = \sqrt{\pv{x-y,x-y}}.
    \]
    \end{remark}
    }
\end{example}

\section{Sobolev spaces of functions}

\subsection{Weak differentiation}

Let $\Omega$ be an open, connected subset of $\RR$. 
Let $\phi$ be a differentiable 
function on $[a,b]$ such that $\phi(a)=\phi(b) = 0$. 
Then by integration by parts, 
\[
    \int_a^b f\phi'dx = \left[f(x)\phi(x)\right]_{a}^{b} - \int_a^b f' \phi dx = - \int_a^b f' \phi dx.
\]
Even when $f'$ does not exist, we may be able to show that some function 
can take the place of $f'$. To generalize, let $\Omega$ be an open, connected
subset of $\RR^n$ (we call this a \vocab{domain}). If there exists a function $g_i$ such that 
\[
    \int_\Omega f\partial_i\phi dx = -\int_\Omega g_i\phi dx,
\] 
for all $\phi$ that are continuous, smooth (infinitely differentiable),
and the set of all values $x$ such that $\phi(x)\neq 0$ is compact,
then define the \vocab{weak derivative} as $\partial_if := g_i$.

\begin{definition}
    A \vocab{multi-index} is shorthand for writing partial derivatives.
    A multi-index $\alpha$ is a tuple of numbers that correspond to how many 
    times each dimension is partially differentiated.
    $\pa{\alpha}$ gives the order of the partial derivative.
\end{definition}

For higher weak derivatives, if the integrable function $u$ has 
\[
    \int_\Omega u\partial^\alpha\phi  = (-1)^{\pa{\alpha}} \int_\Omega v\phi,
\]
for some integrable function $v$, and for all $\phi$ as defined before,
then it is \vocab{$\alpha$-times weakly differentiable}.

\subsection{Sobolev spaces}

\begin{definition}
    For a domain $\Omega$ in $\RR^n$, 
    the \vocab{Sobolev space $W^{k,p}(\Omega)$} is the space 
    of functions $f\in L^{p}(\Omega)$ 
    that are \textit{weakly} differentiable up to $k$ times. 
    \sidenote{
    \begin{remark}
        When $p=2$, the Sobolev space may be written as $H^k(\Omega)$.
        Since $H^k(\Omega) \subseteq L^2(\Omega)$, there exists an 
        induced inner product on $H^k(\Omega)$.
        This is an example of a \vocab{Hilbert space}, which is an 
        inner product space that is a complete metric space.
    \end{remark}
    }
    Moreover, we require that each of the derivatives also has a finite $L^p$ norm and be continuous.
\end{definition}

Sobolev spaces admit the norm 
\[
    \|f\|_{W^{k,p}} := \left (\sum_{i=0}^k \left \|f^{(i)} \right \|_{L_p}^p \right)^{\frac{1}{p}} = \left (\sum_{i=0}^k \int \left |f^{(i)}(t) \right |^p\,dt \right )^{\frac{1}{p}}.
\]

\subsection{The k-extension property}
While there is a general definition for topological vector spaces,
we examine the specific definition for $X,Y$ as normed vector spaces.

\begin{definition}
    A linear operator $L:X\to Y$ is \vocab{bounded} if 
    there exists $M>0$ such that for all $x\in X$,
    \[
        \pn{Lx}_{Y} \leq M\pn{x}_X.
    \]
\end{definition}

\begin{definition}
    If there exists a bounded linear operator 
    $E:W^{k,2}(\Omega) \to W^{k,2}(\RR^n)$ 
    such that $Eu$ with its domain restricted to $\Omega$ equals $u$ 
    for all functions $u$ in 
    $W^{k,2}(\Omega)$, then $\Omega$ satisfies the 
    \vocab{$k$-extension property}.
\end{definition}

The $k$-extension property is a classification of 
domains that is a necessity for
G\r{a}rding's inequality and many other results in PDE analysis.

\section{Differential Operators}

\subsection{Elliptic operators}

\begin{definition}
    Let $\alpha$ represent a multi-index.
    Let $L$ be a partial differential operator, defined for a function 
    $u$ with domain $\Omega$ in $\RR^n$. 
    \[
    Lu = \sum_{\pa{\alpha}\leq m} a_\alpha(x)\partial^\alpha u.
    \]
    $L$ is \vocab{elliptic} if for every $x\in \Omega$ and every non-zero 
    $\xi \in \RR^n$, 
    \[
        (-1)^k\sum_{\pa{\alpha}=2k} a_\alpha(x)\xi^\alpha > 0
    \]
    It can be shown that the order of the PDE must be even to be elliptic 
    ($m=2k$). We then make a stronger condition, \vocab{uniform 
    ellipticity}, for an operator of order $2k$:
    \[
        (-1)^k \sum_{\pa{\alpha}=2k} a_\alpha(x)\xi^\alpha > C\pa{\xi}^{2k}.
    \]
    for a positive constant $C$ and for every $\xi\in\RR^n$.
\end{definition}

\subsection{Induced bilinear forms}

Differential operators induce their own bilinear form.
In order to construct a bilinear form for a differential operator $L$,
we perform integration by parts on $L$:
\[
    \int_\Omega \phi Ludx = \sum_{0\leq \pa{\sigma},\pa{\gamma}\leq k}\int_\Omega a_{\sigma \gamma}(x)\partial^\gamma u \partial^\sigma \phi dx,
\]
for mutli-indices $\gamma$ and $\sigma$.
We define the induced bilinear form by $L$ as 
\[
    B[u,v] := \sum_{0\leq \pa{\sigma},\pa{\gamma}\leq k}\int_\Omega a_{\sigma \gamma}(x)\partial^\gamma u \partial^\sigma v dx
\]

% $u\mid_\Omega$, $\partial \Omega$,
% $\Omega \subseteq \RR^n$

\section{G\r{a}rding's inequality}

\subsection{Inequality statement}

\begin{theorem}
    [G\r{a}rding's inequality]
    \label{thm:gardinginequality}
    Let $\Omega \subseteq \RR^n$ be a bounded domain that 
    has the $k$-extension property.
    Let $u$ be a function in $W^{k,2}(\Omega)$ and 
    $L$ be a uniformly elliptic differential operator 
    of order $2k$, written as
    \[
        (Lu)(x)=\sum _{{0\leq |\alpha |,|\beta |\leq k}}(-1)^{{|\alpha |}}\partial^{{\alpha }}\left(A_{{\alpha \beta }}(x)\partial^{{\beta }}u(x)\right).
    \]
    If $\pa{\alpha} = \pa{\beta} = k$, then require $A_{\alpha\beta}$
    to be bounded and continuous on the closure of $\Omega$. 
    If $\pa{\alpha},\pa{\beta}\leq k$, then require 
    $A_{\alpha\beta}\in L^{\infty}(\Omega)$.
    If all of the above are satisfied, then 
    there exists constants $C>0$ and $G\geq 0$ such that
    \[
        B[u,u]+G\|u\|_{{L^{{2}}}}^{{2}}\geq C\|u\|_{{W^{k,2}}}^{{2}}{\text{ for all }}u\in W^{k,2}_0(\Omega),
    \]
    where $B[u,v]$ is a bilinear form induced by the differential operator 
    $L$, and $W^{k,2}_0(\Omega)$ is the space of functions $u\in W^{k,2}(\Omega)$ such that $u$ evaluates to $0$ on the boundary of $\Omega$.
\end{theorem}

G\r{a}rding's inequality relates
the $L^2$ and Sobolev norm of a function $u$ with the induced bilinear form 
of a uniformly elliptic differential operator.

\subsection{Application: A Dirichlet boundary problem}

A Dirichlet boundary problem is a common type of 
elliptic boundary-value problem.
Consider a bounded region $\Omega \subseteq \RR^n$, a function $f$, and an elliptic operator $L$ of order $2k$. The solution to a Dirichlet boundary problem we will be studying 
is a function $u(x,y)$ that satisfies 
\[
\begin{cases}
    L u = f &\text{ in }\Omega \\
    u = 0 &\text{ on }\partial \Omega.
\end{cases}
\]
Here $\partial \Omega$ denotes the \textit{boundary} of $\Omega$:
the closure minus the interior of a set.

There are three levels of solutions to the Dirichlet problem to discuss.
First are the \vocab{classical solutions}, which require 
the function $u$ to be continuous, $2k$ times differentiable, 
and all of its derivatives to also be continuous.

Lightening our criterion to allow for weak derivatives, 
there are \vocab{strong solutions}, which loosen the requirement 
of $u$ to being $\sim 2k$ times \textit{weakly} differentiable. 

Finally, $u$ is a \vocab{weak solution}
of this Dirichlet problem if $B[v,u] = \int_{\Omega}fvdx$ for every
 $v\in W_0^{k,2}(\Omega)$.
Even fewer weak derivatives are required, 
effectively reducing the ``smoothness'' needed for a function 
to be considered a solution.

As a general hierarchy, 
\[
    \ps{\text{classical sol'ns}} \subseteq \ps{\text{strong sol'ns}} \subseteq \ps{\text{weak sol'ns}}.
\]

Since our classification of weak solutions to the boundary problem requires 
the induced bilinear form $B$, it should be clear that the G\r{a}rding 
inequality is useful here.
Indeed, the G\r{a}rding inequality is used to show the existence of weak solutions to this Dirichlet problem.
\begin{theorem}
    \label{thm:uniqueweaksolns}
    Let $L$ be a differential operator of order $2k$ that satisfies the 
    conditions of \ref{thm:gardinginequality}.
    Then there exists $\lambda \geq 0$ such that 
    for $\lambda_0 \geq \lambda$, the Dirichlet problem for 
    the operator $L+\lambda_0$ has a unique weak solution.
\end{theorem}

\begin{remark}
    The proof for \ref{thm:uniqueweaksolns} requires both the G\r{a}rding 
    inequality and the \textit{Lax–Milgram lemma}, a result that 
    guarantees the existence of unique weak solutions to the Dirichlet problem.
    This result was proven in \cite{LaxMilgram+1955+167+190}.
\end{remark}
\clearpage
\bibliographystyle{alpha}
\bibliography{honors_refs}
\end{document}