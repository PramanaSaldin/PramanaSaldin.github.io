\documentclass[11pt]{scrartcl}
\usepackage[sexy]{pramananotes}
\usepackage[margin=1.5in]{geometry}
\usepackage[utf8]{inputenc}
\numberwithin{equation}{section}

\title{MATH 521 - Analysis I Notes}
\author{Pramana}
\date{Fall 2022}
\begin{document}
\maketitle
\begin{abstract}
    The real numbers, elements of set theory, metric spaces and basic topology, sequences and series, limits, continuity, differentiation, integration, sequences and series of functions, uniform convergence.    
\end{abstract}
\tableofcontents
\clearpage
\section{September 8th, 2022}
Office hours: Tuesday 9-10PM (Zoom), Thursday 2:30-3:30PM (in-person), 725 Van Vleck.
\subsection{The natural numbers}
\begin{axiom}
    [Peano axioms]
    The Peano axioms are:
    \begin{enumerate}[N1.]
        \item $1 \in \NN$.
        \item If $n\in\NN$ then its successor $n+1\in\NN$.
        \item $1$ is not the successor of any element in $\NN$. 
        \item If $n$ and $m$ have the same successor, then $n = m$.
        \item A subset of $\NN$ which contains $1$, and $n + 1$ whenever it contains $n$ must equal to $\NN$.
    \end{enumerate}
\end{axiom}
\begin{claim}
    \label{cl:n5necessary}
    N5 is necessary.
\end{claim}
\begin{subproof}
Suppose that N5 is false. Then $\exists S \subset \NN$ s.t. $1 \in S$, and if 
$n \in S \implies n+1\in S$, but $S \neq \NN$. Now consider the set $T$:
\[ T = \left\{n \in \NN\mid n\notin S\right\},\]
call $n_0$ the smallest number in $T$. Since $n_0\neq1$, $n_0$ is the successor to some $n_0-1$, but $n_0-1\in S$, since it is smaller than $n_0$. But its successor is in $S$ too.
\end{subproof}
\subsection{Addition and multiplication in the rationals}
We need a set $F$ with a group structure.
\begin{enumerate}
    \item[A0] $\forall a,b, a+b\in F$
    \item[A1] $a+(b+c) = (a+b)+c\forall a,b,c$
    \item[A2] $a+b=b+a$
    \item[A3] $a+0=a$
    \item[A4] $\forall a, \ \exists-a$ s.t. $a + (-a) = 0$.
\end{enumerate}
This suffices for a commutative group. Then the multiplication properties:
\begin{enumerate}[1.]
    \item[M0] A0 for $\times$
    \item[M1] A1 for $\times$
    \item[M2] A2 for $\times$
    \item[M3] $a\cdot1 = a$
    \item[M4] $\forall a\neq0\exists a^{-1} s.t. aa^{-1}=1$.
\end{enumerate}
Along with the distributive law: $a(b+c) = ab +ac$ creates a field structure. 
For example, $\QQ$, the set of rational numbers is a field.
\subsubsection{Finding root 2}
If we want a number $d$ s.t. $d\cdot d = 2$. In $\QQ$ we can get close... 
\begin{align*}
    1.4142^2 &= 1.99996164\dots \\
    1.4143^2 &= 2.00024449\dots
\end{align*}
however it isn't possible, since
\begin{claim}
    $\sqrt2$ is not rational.
\end{claim}
\begin{subproof}
    Suppose $\sqrt2 = \frac{p}{q}, p \in \ZZ, q \in \NN$ and $(p,q) = 1$.
    \begin{align*}
        \sqrt{2} &= \frac{p}{q} \\
        2 &= \frac{p^2}{q^2} \\
        2q^2 &= p^2,
    \end{align*}
    which implies that $p$ is even. Then $p = 2k$ for some $k \in \ZZ$.
    This means that $2k^2 = q^2$, so then $q$ is even as well. But that contradicts the assumption that $(p,q)=1$.
\end{subproof}
\begin{definition}
    A number is \vocab{algebraic} if it satisfies a polynomial
    \[ a_nx^n + a_{n-1}x^{n-1}+\cdots+a_1x+a_0 \quad a_i \in \ZZ, \ a_n \neq 0, \ n \geq 1.\]
\end{definition}
$\sqrt2$ satisfies $x^2- 2=0$, therefore it is algebraic.
\clearpage
\section{September 13th, 2022}
\subsection{Rational zeros theorem}
\begin{theorem}
[Rational zeros theorem]
    \label{thm:ratzthm}
    If $r=\frac{p}{q}$, where $(p,q)=1$, and $q \neq 0$, and $r$ satisfies a polynomial, then $q \mid a_n$ and $p \mid a_0$.
\end{theorem}
\begin{proof}
    Substitute in the solution $\frac{p}{q}$:
    \begin{align*}
        a_n \left(\frac{p}{q}\right)^n + a_{n-1} \left(\frac{p}{q}\right)^{n-1}+\cdots+a_0 &=0 \\
        a_n p^n + a_{n-1} p^{n-1}q+\cdots+a_0q^n &=0 \\
        a_np^n &= -q\underbrace{[\cdots]}_{\text{integer}}
    \end{align*}
    Since $(p,q)=1$, we know that $q\mid a_n$.
    Similarly, 
    \[ a_0q^n = -p\underbrace{[\cdots]}_{\text{integer}}.\qedhere\]
\end{proof}
\begin{example}
    [Golden ratio]
    The polynomial with the golden 
    ratio as a root is $f(x) = x^2-x-1$. Can the golden ratio be rational? If so, then $r=\frac{p}{q}$ and $p\mid a_0$ and 
    $q\mid a_n$, or $q\mid1, p\mid1$. So $p=\pm1,q=\pm1$, or $r=\pm1$.
    We verify none satisfy $f$, so the golden ratio is not rational.
\end{example}
\subsection{Order structure}

\begin{axiom}
    [Order axioms]
    We create an order on the set ``$\leq/\geq$'' such that $\forall a,b,c\in F$,
    \begin{itemize}
        \item[O1] Either $a\geq b$ or $b\geq a$.
        \item[O2] If $a\geq b$ and $b\geq a$, then $a=b$.
        \item[O3] If $a\leq b$ and $b\leq c$, then $a \leq c$.
        \item[O4] If $a\leq b$, then $a+c\leq b+c$.
        \item[O5] If $a\leq b$ and $0\leq c$, then $ac \leq bc$. 
    \end{itemize}
    We let $a<b$ be defined as as $a\leq b$ and $a\neq b$.
\end{axiom}

\begin{example}
    [Axiom results]
    We can use these axioms to prove things we take for granted 
    in ordered fields.
    \begin{enumerate}
        \item $a+c = b+c \implies a=b$. \\
        Using A4, $\exists(-c)$ s.t. $c+(-c)=0$ 
        \begin{align*}
            a + c + (-c) &= b + c + (-c)  \\
            a + (c+(-c)) &= b + (c +(-c)) \\
            a + 0 &= b+ 0 \\
            a &= b
        \end{align*}
        \item $a\cdot0=0$.
        \begin{align*}
            a \cdot 0 &= a\cdot(0 + 0) \tag{A3}\\
            &= a\cdot0 + a\cdot0 \tag{DL}\\
            0 &= a\cdot 0.
        \end{align*}
        \item $(-a)b=-ab$.
        \begin{align*}
            a+(-a) &= 0\\
            ab + (-a)b &= (a+(-a))b \tag{DL} \\
            &= 0 \cdot b \\
            &= 0 \\
            (-a)b &=-ab
        \end{align*}
        \item $(-a)(-b) = ab$ 
        \begin{align*}
            (-a)(-b) + (-ab) &= (-a)(-b)+(-a)b \\
            &= (-a)(-b+b) \tag{DL} \\
            &= (-a)(0) \\
            &= 0 
        \end{align*}
        Since $ab + (-ab)=0$, then $(-a)(-b) = ab$. 
        \item If $0<a$, then $0<a^{-1}$.\\
        Suppose that $0<a$, but $a^{-1}\leq 0$. Then
        \begin{align*}
            (a^{-1})+(-a^{-1}) &\leq 0 + (-a^{-1}) \tag{O4} \\
            -a^{-1}&\geq 0
        \end{align*}
        Then $0 \leq a \text{ and } 0 \leq -a^{-1}$
        \begin{align*}
            0&\leq a\cdot-a^{-1} = -1\\
            0&\leq -1,
        \end{align*}
        contradiction.
    \end{enumerate}
\end{example}
Absolute value is defined how you would expect. 
Note that $\pa{a} \geq 0$, $\pa{ab} = \pa{a}\pa{b}$.
\begin{theorem}
    [Triangle inequality]
    \label{thm:tri_ineq}
    If $a=A-B$, and $b=B-C$,
    \[ \pa{a+b} = \pa{A-C} \implies \pa{A-C} \leq \pa{A-B}+\pa{B-C},\]
    or 
    \[\mathrm{dist}(A,C) \leq \mathrm{dist}(A,B) + \mathrm{dist}(B,C).\]
\end{theorem}

\begin{proof}
    \[ -\pa{a} \leq a \leq \pa{a},\quad -\pa{b}\leq b\leq \pa{b},\]
    implies
    \[ -\pa{a}-\pa{b}\leq -\pa{a}+b\leq a+b \leq \pa{a}+b\leq \pa{a}+\pa{b}, \]
    so 
    \[ -\pa{a}-\pa{b}\leq a+b \leq \pa{a}+\pa{b} \implies \pa{a+b} \leq \pa{a}+\pa{b}.  \qedhere\]
\end{proof}

\subsection{Intervals}
\begin{definition}
    Let $S$ be non-empty subset of $\RR$. If there exists $s_0\in S$ 
    such that $s\leq s_0$ for all $s\in S$, then $s_0$ is called the 
    \vocab{maximum} of $S$, denoted $s_0 = \max{S}$.
    Similarly, the smallest element it is the 
    \vocab{minimum} of $S$, denoted $\min{S}$.
\end{definition}
\begin{definition}
    Finite subsets always have a maximum and minimum.
    Suppose $a < b$. Then the \vocab{open interval} between $a$ and $b$ is 
    \[ (a,b) = \left\{x\in\RR \mid a <x<b\right\}. \]
    The \vocab{closed interval} is 
    \[ [a,b] = \left\{x\in\RR \mid a\leq x\leq b\right\}. \]
    A \vocab{semi-open interval} is $(a,b]$.
\end{definition}
Note that $\min[a,b] = a$, $\max[a,b] = b$.
\begin{proposition}
    [Open intervals have no max nor min]
    The open interval $(a,b)$ does not have a maximum nor minimum.
\end{proposition}
\begin{proof}
    For an interval $(0,1)$, suppose that $\lambda=\max(0,1)$. So $\lambda\geq x$ and $1 > x$ $\forall x\in(0,1)$. Consider $1>\frac{\lambda+1}{2} > \lambda$, so 
    we have found a higher "maximum".
    Similar argument for no minimum, and works for any open interval $(a,b)$.
\end{proof}
\clearpage
\section{September 15th, 2022}
\subsection{Least upper bound and greatest lower bound}
\begin{definition}
    $S$ is \vocab{bounded} (above) if $\exists x$ s.t. $x\geq s \ \forall s \in S$.
\end{definition}
\begin{definition}
    Let $S$ be a non-empty subset of $\RR$.
    If $S$ is bounded above and $S$ has a least upper bound, then 
    we call it the \vocab{supremum} of $S$, $\sup{S}$. \\
    If $S$ is bounded below and $S$ has a greatest lower bound, then 
    we call it the \vocab{infimum} of $S$, $\inf{S}$.
\end{definition}
Note that if $\max S$ exists, then $\max S = \sup S$. Also $\sup[a,b] = \sup(a,b) = b$.
\begin{example}
    [Sample sup/inf proof]
    $\sup(0,1) = 1$. 
    $1 \geq x$ for $x \in (0,1)$. Suppose that $t$ is an upper bound, for 
    $(0,1)$ and $t<1$. Consider $x = \frac{1+t}{2}$, then $x \in (0,1)$,
    but $x>t$.
\end{example}
\subsection{The real numbers}
\begin{axiom}
    [Completeness axiom]
    Every non-empty subset $S$ of $\RR$ that is bounded above has 
    $\sup S \in \RR$.
\end{axiom}
\begin{corollary}
    [Completeness axiom below]
    Every non-empty subset $S$ of $\RR$ that is bounded below has 
    a greatest lower bound $\inf S$.
\end{corollary}
\begin{proof}
    Define $-S = \left\{-s \mid s \in S\right\}$. Since $S$ is bounded below,
    $\exists m \in \RR$ s.t. $m\leq s \ \forall s \in S$.
    Then $-m\geq -s \ \forall s \in S$. Therefore $-m \geq u \ \forall u \in -S$, showing the existence of $\sup -S$.

    I claim $\inf S = -s_0$. 
    \begin{itemize}
        \item First we show that $s_0$ is a lower bound. 
        For all $s\in S$, $-s\in -S$, so 
        \[
            -s\leq s_0 \implies s\geq -s_0.
        \]

        \item Then we show for any other lower bound $t$, $t\leq -s_0$. 
        This implies that $-t\geq -s \ \forall s\in S$, or $-t\geq x \forall x \in -S$, so $-t$ is a upper bound for $-S$.
        So 
        \[ 
            -t\geq\sup(-S) = s_0 \implies -t\geq s_0 \implies t \leq -s_0. 
        \]
    \end{itemize}
    Thus, $\inf S = -s_0$.
\end{proof}
\begin{proposition}
[Archimedian property]
    \label{prop:archprop}
    If $a>0$ and $b>0$, then $\exists n\in\NN$ s.t. $na > b$.
\end{proposition}
\begin{proof}
    (intuitive)
    a bathtub and a spoon
\end{proof}
\begin{proof}
    Assume that it fails. So $\exists a>0, b>0$ such that 
    $b$ is an upper bound for the set 
    $S = \left\{an \mid n \in \NN\right\}$.
    Let $s_0 = \sup S$, which must exist by the Axiom of completeness.
    $a>0 \implies s_0-a<s_0$. Since $s_0$ is the least upper bound, $s_0-a$
    can't be an upper bound. Meaning there is some $n_0\in\NN$ s.t. 
    $s_0-a<n_0a$. Therefore $s_0 <(n_0+1)a$, but 
    $(n_0+1)a \in S$, so $s_0$ is not an upper bound.
\end{proof}
\begin{proposition}
    [Denseness of $\QQ$]
    \label{prop:denseq}
    If $a,b\in\RR$, $\exists r\in\QQ$ s.t. $a<r<b$.
\end{proposition}
\begin{proof}
    Show there is $a<\frac{m}{n}<b$ for $m,n\in\ZZ, n\neq 0 \iff na < m <nb$.
    We have that $b-a>0$, so \cref{prop:archprop} says that $\exists n$ s.t.
    $n(b-a) >1$. So $\exists k \in \NN$ s.t. $k>\max \left\{\pa{an},\pa{bn}\right\}$, so that 
    \[ -k <an <bn < k.\]
    Consider $S = \left\{j\in\ZZ\mid-k\leq j\leq k, an < j\right\}$,
    and look at $m = \min S$.
    So $m-1\leq an$ and
    \[ m = (m-1) + 1\leq an+1<an+(bn+an) = bn. \qedhere \]
\end{proof}
\subsection{Infinity}
$\infty$ and $-\infty$ are useful \textit{symbols}, 
but they are not in $\RR$.
We can extend ordering to $\RR\cup \left\{-\infty, \infty\right\}$ 
by saying $-\infty \leq a \leq \infty \ \forall a\in \RR$.
\begin{example}
    [Unbounded intervals]
    Consider 
    \[ [a,\infty) = \left\{x\in\RR\mid a\leq x\right\}\]
    \[ (a,\infty) = \left\{x\in\RR\mid a< x\right\}.\]
\end{example}

Moreover, $\sup S= \infty$ if $S$ is not bounded above, and 
$\inf S = -\infty$ is $S$ is not bounded below.

\subsection{Sequences and convergence}
\begin{definition}
    A \vocab{sequence} is a function from 
    $\left\{n \in\ZZ\mid n\geq m\right\}$ to $\RR$.
    Usually $m=0,1$. We denote sequences as 
    \[ \left(s_1,s_2,\dots\right) \text{ or } (s_n)_{n\in\NN}.\]
\end{definition}
\begin{definition}
    $s_n$ \vocab{converges} to $s$ if $\forall\varepsilon > 0$,
    $\exists N$ such that $n>N \implies \pa{s_n-s} < \varepsilon$.
    If this is satisfied, write 
    \[ 
        \lim_{n\rightarrow\infty} s_n = s. 
    \]
    $N$ can be forced to be an integer.
    $\varepsilon$ can be any positive real number, but typically we use 
    it for cases where it is small.
\end{definition}
\begin{proposition}
    Limits are unique. 
\end{proposition}
\begin{proof}
    Suppose $a_n \rightarrow a$ and $a_n\rightarrow b$, but $b\neq a$.
    WLOG $a<b$, let $d=b-a$. Let 
    $\varepsilon = \frac{d}{3}$. Thus $\exists N_1$ s.t. 
    $\forall n>N_1, \pa{a_n-a} < \frac{d}{3}$, and $\exists N_2$ s.t. 
    $\forall n>N_2, \pa{a_n-b} < \frac{d}{3}$, but then 
    \begin{align*}
        \pa{a-b} &= \pa{(a-a_n)+ (a_n-b)} \\
        &\leq  \pa{a-a_n} + \pa{a_n-b} \\ 
        &< \frac{2d}{3}, 
    \end{align*}
    which is a contradiction.
\end{proof}
\clearpage
\section{September 20th, 2022}
\subsection{Divergence}
A sequence that does not converge \vocab{diverges}.
\begin{example}
    $s_n = (-1)^n$ diverges.
    \begin{proof}
        Suppose that $s_n\rightarrow a$. Then 
        $\exists N$ s.t. $n>N\implies \pa{s_n-a}<\half$.
        Choose $n_1$ even and $n_2$ odd. $n_1,n_2>N$.
    
        Note $2=\pa{s_{n_1} - s_{n_2}} = \pa{(s_{n_1} - a) - (s_{n_2} - a)}
        \leq \pa{(s_{n_1} - a)} + \pa{(s_{n_2} - a)} < \half + \half = 1$,
        which is a contradiction.
    \end{proof}
\end{example}

\begin{example}
    Let $(s_n)$ be a convergent seq. s.t. $s_n\neq 0 \ \forall n \in \NN$,
    and 
    \[ \lim_{n\rightarrow \infty}s_n = s \neq 0.\] 
    Then $\inf \left\{\pa{s_n}\mid n \in \NN\right\}>0$.

    Choose $\eps$ so that seq. will lie in a region $\eps$ from 
    $0$. Let $\eps = \frac{\pa{s}}{2}$. 
    There exists $N$ s.t. 
    $n>N \implies \pa{s_n-s}<\eps$.
    \begin{align*}
        \pa{s} &= \pa{s_n+s-s_n} \\
        &\leq \pa{s_n} + \pa{s-s_n}.\\
        \implies \pa{s_n} &\geq \pa{s}-\pa{s-s_n} \\
        &> \pa{s}-\frac{\pa{s}}{2}\\ 
        &= \frac{\pa{s}}{2}.
    \end{align*}
    Let $m =\min \left\{\frac{\pa{s}}{2}, \pa{s_1},\pa{s_2},\dots, \pa{s_N}\right\}$. $\pa{s_n} \geq m$ for all $m\in \NN$.
    Therefore $m$ is a lower bound for $\inf\left\{\pa{s_n}\mid n \in \NN\right\}\geq m > 0$.
\end{example}
\begin{theorem}
    [Sandwich lemma]
    \label{thm:sandle}
    Suppose that $(a_n),(b_n),(s_n)$ are sequences so that 
    \[ a_n\leq s_n \leq b_n \ \forall n. \]
    Suppose that $s=\lim_{n\rightarrow \infty} a_n = \lim_{n\rightarrow \infty} b_n.$ Then $\lim_{n\rightarrow \infty} s_n =s$. 
\end{theorem}
\begin{proof}
    We want to show that $s-\eps<s_n<s+\eps$ eventually.
    
    Choose $\eps>0$, then $\exists N_1$ s.t. 
    $n>N_1 \implies \pa{a_n-s}<\eps \implies a_n>s-\eps$.
    Choose $N_2$ for $b$ as well, so that
    $n>N_2\implies b_n<s+\eps$. Choose $n>N_1$ and $n > N_2$. Then 
    \[ s-\eps < a_n \leq s_n \leq b_n < s+ \eps 
    \implies \pa{s_n-s} < \eps. \qedhere \]
\end{proof}
\begin{proposition}
    Convergent sequences are bounded.
\end{proposition}
\begin{proof}
    Let $(s_n)_{n\in\NN}$ have $\lim_{n\rightarrow\infty =s}$.
    Then $\exists N$ s.t. $n>N\implies \pa{s_n-s}<1$.
    By the triangle inequality,
    \begin{align*}
        \pa{s_n} &\leq \pa{s}+\pa{s_n-s} \\
        &<\pa{s}+1.
    \end{align*}
    Choose $M = \max \left\{\pa{s}+1, \pa{s_1},\pa{s_2},\dots\right\}.$
    Then $\pa{s_n}\leq M\forall n \in \NN$, so $(s_n)$ is bounded.
\end{proof}
\subsection{Useful limit properties}
\begin{proposition}
    [Scalar limits]
    If $s_n\rightarrow s$  and $k \in \RR$,  then $ks_n \rightarrow ks$
    as $n\rightarrow\infty$.
\end{proposition}
\begin{proof}
    If $k=0$, then $ks_n = 0$, which is immediately true.
    
    If $k\neq 0$, then choose $\eps >0$. Then $\exists N$ s.t. 
    $n>N \implies \pa{s_n-s}<\frac{\eps}{\pa{k}} \implies 
    \pa{k_{s_n}-ks} < \eps.$
\end{proof}
\begin{proposition}
    Is $s_n\rightarrow s$, and $t_n\rightarrow t$, then
    \begin{enumerate}
        \item $s_n+t_n \rightarrow s+t$ as $n\rightarrow \infty.$
        \item $s_n\cdot t_n \rightarrow s\cdot t$ as $n\rightarrow \infty.$
        \item if $s_n\rightarrow s$ and $s_n\neq 0 \ \forall n$ and $s\neq0$,
        then $\frac{1}{s_n}\rightarrow \frac{1}{s}$.
    \end{enumerate} 
\end{proposition}
\begin{proof}
    (1) Let $\pa{s_n-s},\pa{t_n-t}<\frac{\eps}{2}$, and the result follows.

    (2) We want $\pa{s_nt_n -st} < \eps$.
    \begin{align*}
        \pa{s_nt_n -st} &= \pa{s_nt_n-s_nt+s_nt-st} \\
        &\leq \pa{s_nt_n-s_nt}+\pa{s_nt-st} \\ 
        &= \pa{s_n}\pa{t_n-t}+\pa{t}\pa{s_n-s}.
    \end{align*}
    Since $s_n\rightarrow s$, it is bounded. So $\pa{s_n}<M$ for some $M$.
    Choose $\eps>0$:
    \begin{itemize}
        \item $\exists N_1$ s.t. $n>N_1\implies \pa{t_n-t}<\frac{\eps}{2M}$
        \item $\exists N_2$ s.t. $n>N_2\implies \pa{s_n-s}<\frac{\eps}{2 (\pa{t} + 1)}$ (since $\pa{t}$ may be $0$).
    \end{itemize}
    Set $N=\max \left\{N_1,N_2\right\}$. For $n>N$,
    \begin{align*}
        \pa{s_nt_n -st} &\leq\pa{s_n}\pa{t_n-t}+\pa{t}\pa{s_n-s} \\
        &< M\cdot\frac{\eps}{2M} +  \frac{\eps}{2(\pa{t}+1)}\pa{t}. \\
        &< \eps.
    \end{align*}
    
    (3) Let $\eps>0$. Since $(s_n)$ is bounded, 
    $\exists m>0$ s.t. $\pa{s_n}\geq m$.
    $\exists N \in \NN$ s.t. for $n>N \implies \pa{s-s_n} < \eps M \pa{s}$.
    \begin{align*}
        \pa{\frac{1}{s_n}-\frac{1}{s}} &= \pa{\frac{s}{s\cdot s_n}-\frac{s_n}{s_n\cdot s}} \\
        &=\frac{\pa{s-s_n}}{\pa{s_n\cdot s}} \\
        &\leq \frac{\pa{s-s_n}}{m\pa{s}} \\
        &< \eps. \qedhere
    \end{align*}
\end{proof}

We can combine all the properties to relate how limits of the 
combination of two sequences will end up.
For example, $\lim_{n\rightarrow\infty} \frac{t_n}{s_n} = \lim_{n\rightarrow\infty} t_n \cdot\frac{1}{s_n} = t \cdot \frac{1}{s} = \frac{t}{s}$.

\clearpage
\section{September 22nd, 2022}
\subsection{Convergent sequences example}

\begin{example}
    $\lim_{n\rightarrow\infty}a^n = 0 \text{ if } \pa{a} < 1$.
    \begin{proof}
        If $a=0$, obvious. Otherwise let $\pa{a}=\frac{1}{1+b}$.
        Note that 
        \[ (1+b)^n = 1+nb+\cdots \geq 1 +nb > nb \text{ for } n\in\NN,\]
        by the binomial theorem. Then 
        \[ \pa{a^n-0}=\pa{a^n}=\frac{1}{(1+b)^n} < \frac{1}{nb}. \]
        For $\eps > 0$, By letting $N=\frac{1}{\eps b}$, then 
        $\pa{a^n-0}<\eps \forall n > N$.
    \end{proof}
\end{example}

\subsection{Divergence to infinity}

\begin{definition}
    [Infinite limits]
    Write that 
    \[ \lim_{n\rightarrow \infty}s_n = \infty \]
    if $\forall M>0$, there exists $N$ s.t. 
    \[ n>N \implies s_n>M. \]
    Write that 
    \[ \lim_{n\rightarrow \infty}s_n = -\infty \]
    if $\forall M<0$, there exists $N$ s.t. 
    \[ n>N \implies s_n<M. \]
    In both of these cases $s_n$ \vocab{diverges to $\pm\infty$}.
\end{definition}

\begin{definition}
    $(s_n)$ has a \vocab{limit} if it converges or diverges to $\pm\infty$.
\end{definition}

\begin{example}
    $\lim_{n\rightarrow\infty}n = \infty$ and 
    $\lim_{n\rightarrow\infty}-n = -\infty$.
\end{example}

\begin{proposition}
    Let $(s_n)$ and $(t_n)$ so that $s_n$ diverges to infinity, and 
    $\lim t_n > 0$.
    Then $\lim s_n t_n = \infty$.
\end{proposition}

\begin{subproof}
    Split into cases:
    \begin{itemize}
        \item \textbf{Case 1: }$\lim t_n=t \in\RR$. Then 
        $\exists N_1$ s.t. $n>N_1\implies \pa{t_n-t}<\frac{t}{2}
        \implies t_n>\frac{t}{2} =: \lambda$.
        \item \textbf{Case 2: }$\lim t_n=\infty$. Then
        $\exists N_1$ s.t. $n>N_1\implies t_n>1 := \lambda$.
    \end{itemize}
    Choose $M>0$, since $s_n\rightarrow \infty$, there exists $N_2$ s.t.
    $n>N_2$ s.t. $s_n >\frac{M}{\lambda}$. Set $N=\max \left\{N_1,N_2\right\}$, then $n>N \implies t_ns_n > \lambda \cdot\frac{M}{\lambda}=M$.
\end{subproof}

\subsection{Monotonic sequences}
\begin{definition}
    A seq. $(s_n)$ is \vocab{non-decreasing} if $s_n\leq s_{n+1} \forall n$,
    and \vocab{non-increasing} if $s_n\geq s_{n+1} \forall n$.

    A seq. that is either non-decreasing or non-increasing is
    \vocab{monotonic}.
\end{definition}
\begin{example}
    [Monotonic sequences]
    $s_n=n$ is an unbounded monotonic sequence. \\
    $s_n=1-\frac{1}{n}$ or $s_n=\frac{1}{n}$ are bounded monotonic sequences.
\end{example}

\begin{theorem}
    \label{thm:bddmonotonicconv}
    All bounded monotonic sequences converge.
\end{theorem}

\begin{proof}
    Let $(s_n)$ be a non-decreasing seq.
    It is easy to show that the limit is $\sup\ps{s_n\mid n\in \NN}$
    by supremum properties.
\end{proof}

\begin{proposition}
    If $s_n$ is an unbounded non-decreasing sequence, then 
    $\lim_{n\rightarrow\infty} s_n = \infty$.
\end{proposition}
\begin{proof}
    $\left\{s_n\mid n\in\NN\right\}$ is bounded below by $s_1$,
    and unbounded above. For any $M>0, \exists N\in\NN$ s.t.
    $s_N>M$. Since the sequence is non-decreasing, 
    $n>N$, $s_n\geq s_N>M$ for all  as well.
\end{proof}
\begin{corollary}
    If $(s_n)$ is monotonic, it either converges, or diverges to $\pm\infty$.
    Thus $\lim s_n$ is always meaningful.
\end{corollary}
\subsection{Liminf and limsup}
\begin{definition}
    For a sequence $(s_n)$, we define associated sequences $(u_N)$, $(v_N)$ as 
    \[ u_N = \inf \left\{s_n\mid n>N\right\}, \]
    \[ v_N = \sup \left\{s_n\mid n>N\right\}. \]
    Then $u_1 \leq u_2\leq u_3 \leq \cdots$, 
    so $(u_N)$ is a non-decreasing sequence, and 
    $v_1 \geq v_2\geq v_3 \geq \cdots$, so $(v_N)$ is a non-increasing seq.
    Define the \vocab{$\limsup$} and \vocab{$\liminf$} as
    \[ 
        \limsup_{n\rightarrow \infty} s_n := \lim_{N\rightarrow\infty}v_N 
        \quad \text{ and }\quad 
        \liminf_{n\rightarrow \infty} s_n := \lim_{N\rightarrow\infty}u_N. 
    \]
    These are useful, since $(u_N)$ and $(v_N)$ are both monotonic,
    therefore 
    $\limsup_{n\rightarrow \infty} s_n$ and 
    $\liminf_{n\rightarrow \infty} s_n$ both exist.
\end{definition}
\begin{example}
    For $s_n=(-1)^n$ and
    \[ u_N = \inf \left\{s_n\mid n>N\right\} \qquad 
    v_N = \sup \left\{s_n\mid n>N\right\}, \] 
    \[ \liminf_{n\rightarrow\infty} s_n = \lim_{N\rightarrow\infty}u_N = -1\qquad 
    \limsup_{n\rightarrow\infty} s_n = \lim_{N\rightarrow\infty}v_N = 1, \]
\end{example}
\begin{theorem}
    For a sequence $(s_n)$,
    \begin{enumerate}
        \item If $\lim s_n$ exists, then $\liminf s_n = \lim s_n =\limsup s_n$,
        \item $\liminf s_n =\limsup s_n \implies \lim s_n$ is defined,
        and $\liminf s_n = \lim s_n =\limsup s_n$.
    \end{enumerate}
\end{theorem}
This can be proven with \cref{thm:sandle}.

\subsection{Cauchy sequences}

\begin{definition}
    A sequence $(s_n)$ is \vocab{Cauchy} if 
    $\forall \eps>0$, $\exists N$ s.t. $m,n>N$, then 
    $\pa{s_n-s_m}<\eps$.
\end{definition}
\clearpage
\section{September 27th, 2022}
\subsection{Cauchy results}

\begin{theorem}
    [Convergence is Cauchy]
    \label{thm:Cauchy1way}
    Convergent sequences are Cauchy sequences.
\end{theorem}

\begin{proof}
    Suppose $\lim s_n=s$. Then 
    \begin{align*}
        \pa{s_n-s_m} &= \pa{s_m-s+s-s_n} \\
        &\leq \pa{s_n-s} + \pa{s_m-s}.
    \end{align*}
    Choose $\eps >0$. $\exists N$ s.t. for $m,n>N$, $\pa{s_m-s},\pa{s_n-s}<\frac{\eps}{2}$,
    so \[ \pa{s_n-s_m} \leq \pa{s_n-s} + \pa{s_m-s} < \frac{\eps}{2} + \frac{\eps}{2} = \eps. \qedhere\]
\end{proof}

\begin{theorem}
    [Cauchy is bounded]
    \label{thm:cauchybdd}
    Cauchy sequences are bounded.
\end{theorem}
\begin{proof}
    Fix $\eps=1$ to find $n \in \NN$ s.t. 
    $m,n > N\implies \pa{s_n-s_m}<1.$ Then $n>N \implies \pa{s_{N+1}-s_n}<1 \implies \pa{s_n} < \pa{s_{N+1}} + 1$.
    
    Let $\pa{s_n}$ is bounded by 
    $M = \max \left\{\pa{s_1},\pa{s_2},\dots,\pa{s_N}, \pa{s_{N+1}+1}\right\}$. 
\end{proof}
\begin{theorem}
[Cauchy is convergence]
    \label{thm:cauchyconv}
    Cauchy sequences are convergent sequences (converse of \cref{thm:Cauchy1way}).
\end{theorem}
\begin{proof}
    Choose $\eps>0$. $\exists N$ s.t. $m,n>N\implies \pa{s_n-s_m}<\eps\implies s_n<s_m +\eps$, so 
    $s_m + \eps$ is upper bound for $\left\{s_n\mid n>N\right\}$.
    Then $v_N = \inf \left\{s_n\mid n>N\right\} \leq s_m+\eps$ for $m > N$.
    Then $v_N-\eps$ is a lower bound for $\sup\left\{s_m\mid m>N\right\}$,
    so $v_N < \sup \left\{s_m\mid m>N\right\} = u_N$.

    Thus,
    \[ \limsup s_n \leq v_N \leq u_N + \eps \leq (\liminf s_n) + \eps \]
    for arbitrarily small $\eps>0$, and $\liminf s_n = \limsup s_n$,
    and the sequence converges.
\end{proof}
\subsection{Subsequences}
\begin{definition}
    A \vocab{subsequence} of seq. $(s_n)_{n\in\NN}$ has the form 
    $(t_k)_{k\in\NN}$ where for each $k$, the is a positive integer 
    $n_k$ so that 
    \[ n_1 < n_2 <\cdots,< n_k<n_{k+1}<\cdots\]
    and $t_k = s_{n_k}$.
    
    A \vocab{selection function} may be defined as 
    $\sigma : \NN \rightarrow \NN: \sigma(k) = n_k \text{ for } k \in \NN$
    so that $t_k = t(k) =  s \circ \sigma (k) = s(\sigma(k)) = s(n_k) = s_{n_k}$.
\end{definition}
\begin{example}
    Let $s_n = n^2(-1)^n : (-1,4,-9,16,-25,\dots)$.
    The positive terms are a subsequence. This is given by 
    $\sigma(k) = n_k=2k$, or $s_{n_k}=(2k)^2(-1)^{2k} = 4k^2$.
\end{example}

\begin{proposition}
    Suppose a sequence $(s_n)$ has $s_n>0\ \forall n \in \NN$,
    and $\inf \left\{s_n\mid n\in \NN\right\} = 0$.
    Then there exists a subsequence $(s_{n_k})$ with limit $0$.
\end{proposition}

\begin{proof}
    Elements are arbitrarily close to $0$ by
    $\inf \left\{s_n\mid n \in\NN\right\} = 0$. 
    We take a subsequence bounded by $\pp{\frac{1}{n}}_{n\in\NN}$ 
    to finish.
\end{proof}

\begin{proposition}
    If $(s_n)$ converges, then every subsequence converges to the 
    same limit.
\end{proposition}

We now prove the Bolzano-Weierstra\ss \ Theorem.

\begin{lemma}
    Every sequence $(s_n)$ has a monotonic subsequence.
\end{lemma}
\begin{proof}
    Define a term $n$ as \textit{dominant} if $s_n>s_m\forall m > n$.
    There are two cases: 
    \begin{enumerate}
        \item \textbf{Case 1:} There are infinite dominant terms.
        Define $(s_{n_k})$ as subsequence of them. Then $s_{n_{k+1}} < s_{n_k}$ for all $k$, so it is a decreasing sequence.
        \item \textbf{Case 2:} There are finitely many dominant terms.
        Choose $n_1$ past all of them. Given $N \geq n_1$, then 
        $\exists m > N$, s.t. $s_m\geq s_N$.
        If we continue choosing terms like this, then we have a 
        non-decreasing sequence. \qedhere
    \end{enumerate}
\end{proof}
\begin{theorem}
[Bolzano-Weierstra\ss \ Theorem]
    \label{thm:bwt}
    Every bounded sequence has a convergent subsequence.
\end{theorem}
\begin{proof}
    The sequence has a monotonic subsequence by the previous lemma,
    and it is bounded. By \cref{thm:bddmonotonicconv} that subsequence 
    converges.
\end{proof}

Visually, we split the bounds of the sequence in half. At least one half 
will have infinitely many terms. We continue this process 
to make a split arbitrarily small region with infinitely many terms.
This expands to $2$ dimensions with splitting into $4$ squares. 

\begin{definition}
    A \vocab{subsequential limit} is any $s\in\ol{\RR}$ 
    that is the limit of some subsequence of $(s_n)$.
\end{definition}

\begin{example}
    Let $s_n = \left( n \sin\frac{\pi n}{2} \right)+ \frac{1}{n}$
    on $n\in\NN$. The subsequential limits are 
    $S = \left\{0,-\infty, \infty\right\}$.
\end{example}

\begin{proposition}
    There is a monotonic subsequence of $(s_n)$ that has limit
    $\limsup s_n$ and $\liminf s_n$.
\end{proposition}

\begin{theorem}
    If $(s_n)$ is a sequence in $\RR$, and $S$ is the set of 
    subsequential limits of $(s_n)$.
    Then 
    \begin{enumerate}
        \item $S$ is non-empty.
        \item $\sup S= \limsup s_n$, and $\inf S= \liminf s_n$
        \item $\lim s_n$ exists $\iff$ $S = \{\lim s_n\}$.
    \end{enumerate}
\end{theorem}
\begin{proof}
    [Proof of (2)]
    Consider a subsequence $(s_{n_k})_{k\in\NN}$ with limit $t$.
    Then 
    \[ t = \liminf s_{n_k} = \limsup s_{n_k}.\]
    In addition, $\left\{s_{n_k}\mid k > N\right\}\subseteq \left\{s_n\mid n > N\right\}$ and 
    \[ 
        \liminf s_n \leq\liminf s_{n_k} = t=\limsup s_{n_k} \leq \limsup s_k.
    \]
    By previous result, there are subsequences that tend to 
    $\liminf s_n, \limsup s_n$, so 
    \[ \inf S = \liminf s_n, \qquad \sup S = \limsup s_n.\qedhere\]
\end{proof}

\clearpage
\section{September 29th, 2022}
\subsection{Subsequence results}

\begin{theorem}
    [S is a closed set]
    \label{thm:subseqlimclosed}
    Let $S$ be the set of subsequential limits of $(s_n)$.
    Suppose $(t_n)$ is a subsequence of $S\cap\RR$, and $t = \lim t_n$.
    Then $t \in S$.
\end{theorem}

\begin{proof}
    Since a subsequence of $(s_n)$ converges to $t_1$, there exists 
    $n_1$ s.t. $\pa{s_{n_1} - t_1} < 1$. Choose $n_1<n_2<\cdots$, s.t. 
    $\pa{s_{n_j} - t_j} < \frac{1}{j}$ for $j =1,2,\dots,k$.

    Suppose $\pa{s_{n_k}-t} \leq \pa{s_{n_k}-t_k} + \pa{t_k-t} < \frac{1}{k}+\pa{t_k-t}$.
    Consider $\eps > 0$. There is $N_1$ s.t. 
    $k>N_1 \implies \pa{t_k-t} < \frac{\eps}{2}$, and
    $N_2$ s.t. $\frac{1}{k}< \frac{\eps}{2}$. 
    Let $N = \max \left\{N_1, N_2\right\}$. 
    \[ \pa{s_{n_k} - t} < \frac{\eps}{2} + \frac{\eps}{2} = \eps. \qedhere\]
\end{proof}

For the next theorem, we allow for $s\cdot\infty = \infty$
and $s \cdot (-\infty) = -\infty$.

\begin{theorem}
    Given sequences $(s_n)$ and $(t_n)$ s.t. 
    $\lim{s_n}=s>0$,
    then 
    $\limsup{s_nt_n} = s (\limsup t_n)$. 
\end{theorem}

\begin{proof}
    Suppose $\limsup t_n$ is finite, and equal to $\beta$.
    Then there is a subsequence $(t_{n_k})$ that converges to $\beta$.
    In addition $\lim_{k\rightarrow \infty}{s_{n_k}} = s$,
    then \[ \lim_{k\rightarrow \infty}{s_{n_k}t_{n_k}} = \beta s.\]
    Therefore $s\limsup t_n = \beta s\leq \limsup s_nt_n$.
    
    To avoid division by $0$, we ignore any finite number of terms of $(s_n)$ and 
    assume $s_n\neq 0$ for all $n$ beyond some point, since 
    $\lim s_n \neq 0$.

    Then $\lim \frac{1}{s_n} = \frac{1}{s}$. 
    \[ \limsup t_n = \limsup \left(\frac{1}{s_n}\right)s_nt_n \geq \frac{1}{s}\limsup s_nt_n. \]
    Therefore $s\limsup t_n \geq \limsup s_nt_n$, which proves equality.
\end{proof}
\begin{theorem}
    Let $(s_n)$ be any sequence of non-zero numbers. 
    Then 
    \[ \liminf \pa{\frac{s_{n+1}}{s_n}} \leq \liminf \pa{s_n}^{\frac{1}{n}}
    \leq \limsup \pa{s_n}^{\frac{1}{n}}\leq \limsup \pa{\frac{s_{n+1}}{s_n}}.\]
\end{theorem}
\begin{proof}
    [Proof of third inequality]
    Let $\alpha = \pa{s_n}^{\frac{1}{n}}$, and 
    $L = \pa{\frac{s_{n+1}}{s_n}}$.
    It suffices to show that $\alpha \leq L_1 \forall L_1>L.$
    \begin{align*}
        L &= \limsup \pa{\frac{s_{n+1}}{s_n}} \\
        &= \limsup_{N\rightarrow \infty}\left\{\pa{\frac{s_{n+1}}{s_n}} : n > N\right\} < L_1.
    \end{align*}
    Where $L_1$ is arbitrarily larger than $L$. 
    Then there exists $N\in\NN$ s.t. 
    \[ \sup \left\{\pa{\frac{s_{n+1}}{s_n}} : n > N\right\} < L_1,\]
    and therefore 
    \[ \pa{\frac{s_{n+1}}{s_n}} < L_1 \text{ for } n>N.\]
    For $n>N$, 
    \[ \pa{s_n} = \pa{\frac{s_{n}}{s_{n-1}}}\pa{\frac{s_{n-1}}{s_{n-2}}}
    \cdots \pa{\frac{s_{N+1}}{s_N}}\pa{s_N} < L_1^{n-N}\pa{s_N}.\]
    Since $L_1$ and $N$ are fixed, let $a = L_1^{-N}\pa{s_N}$.
    \begin{align*}
        \pa{s_n} &< L_1^na \\
        \pa{s_n}^{\frac{1}{n}} &< L_1a^{\frac{1}{n}} \text{ for } n > N.
    \end{align*}
    We have that $\lim_{n\rightarrow \infty}{a^{\frac{1}{n}}} =1$.
    Hence $\alpha= \limsup \pa{s_n}^{\frac{1}{n}}\leq L_1,$
    so $\alpha \leq L$.
\end{proof}

\begin{corollary}
    If $\lim_{n\rightarrow\infty} \pa{\frac{s_{n+1}}{s_n}}$ exists and is $L$, then 
    $\lim_{n\rightarrow\infty} \pa{{s_n}^{\frac{1}{n}}}$ exists and is $L$.
\end{corollary}

\subsection{Series}

When we write $\sum_{n=m}^{\infty}a_n$,
we consider the sequence of partial sums 
\[ s_n = \sum_{k=m}^{n}a_k.\]
The \vocab{infinite series} $\sum_{n=m}^{\infty}a_n$, \vocab{converges}
if the sequence of partial sums converges.
\[ 
    \sum_{n=m}^{\infty}a_n := \lim_{n\rightarrow \infty}{\left(\sum_{k=m}^{n}a_k\right)} = \lim_{n\rightarrow \infty}{s_n}= S.
\]
By making the ordering stay consistent across the 
partial sums, we avoid different convergences.
$\sum_{n=m}^{\infty}a_n$ \vocab{diverges to $\pm\infty$} provided 
that $\lim_{n\rightarrow \infty}{s_n} = \pm\infty$.
We can write $\sum a_n$ to refer to the sum of the series in general.

\begin{example}
    [Series examples]
    Some examples of series from class
    \begin{itemize}
        \item Consider the sum
        \[ \sum_{k=0}^{n}a^n = 1+a+a^2+\cdots+a^n= \frac{a^{n+1}-1}{a-1}.\]
        Since we showed that $\lim_{n\rightarrow \infty}{a_n} = 0$ 
        if $\pa{a}<1$ in which case 
        \[ \sum_{k=0}^{\infty}a^n = \lim_{n\rightarrow \infty}\frac{a^{n+1}-1}{a-1} = \frac{1}{1-a}.\]
        \item The sum
        \[ \sum_{k=1}^{\infty}\frac{1}{k^p}\]
        converges if and only if $p>1$.
        When $p=2$, the sum is $\frac{\pi^2}{6}$, $p=3$ yields 
        $1.2020569$ (Ap\'ery's constant).
    \end{itemize}
\end{example}

\begin{remark}
    [Riemann zeta function]
    The Riemann zeta function is 
    \[ \zeta(p) = \sum_{k=1}^{\infty}\frac{1}{k^p},\]
    and is defied for real $p>1$. Can be analytically 
    extended to $\CC$.
    There are zeros of this function at $-2,-4,-6,\dots$, but they 
    are "trivial".

    There are also sporadic zeros on the line $\Re(p) = \frac{1}{2}$.
    The Riemann hypothesis says that all zeros lie on this line.
\end{remark}

\begin{proposition}
    $a_k$ is Cauchy if
    $\sum a_k$ satisfies $\forall \eps>0, \exists N$ s.t.
    $n \geq m>N$ implies \[ \pa{s_{m-1}-s_n} < 
    \eps,\]
    or 
    \[
        \pa{\sum_{k=m}^{n}a_k} < \eps. 
    \]
\end{proposition}

\begin{corollary}
    If $\sum a_k$ converges, then $\lim a_n=0$.
\end{corollary}

\begin{proposition}
    [Comparison test]
    \label{prop:comptest}
    Let $\sum a_n$ be a series where $a_n\geq 0\forall n$. Then, 
    \begin{enumerate}
        \item If $\sum a_n$ converges  and $\pa{b_n}\leq a_n$, then 
        $\sum b_n$ converges.
        \item If $\sum a_n=\infty$ diverges and $\pa{b_n}\geq a_n$, then 
        $\sum b_n = \infty$.
    \end{enumerate}
\end{proposition}
\begin{proof}
    (1) Consider the Cauchy criterion,
    \[ \pa{\sum_{k=m}^{n}b_k} \leq \sum_{k=m}^{n}\pa{b_k} 
    \leq \sum_{k=m}^{n}a_k.\]
    For all $\eps>0$, there exists $N$ s.t. $n\geq m > N$ implies 
    \[ \pa{\sum_{k=m}^{n}b_k} \leq \pa{\sum_{k=m}^{n}a_n}< \eps.\]

    (2) We know that 
    \[ \sum_{k=1}^{n}a_k \leq \sum_{k=1}^{n}b_k.\]
    Then $\forall M>0$, $\exists N$ s.t. $n>N$ implies
    \[ M < \sum_{k=1}^{n} a_k \leq \sum_{k=1}^{n} b_k  \implies \sum b_n = \infty.
    \qedhere \]
\end{proof}
\begin{definition}
    $\sum a_k$ is \vocab{absolutely convergent} if $\sum \pa{a_k}$ converges.
    Note that $\sum \pa{a_k}$ is monotonic and always has a meaningful
    limit. Absolutely convergent $\implies$ convergent.
\end{definition}
\begin{proposition}
    [Ratio test]
    \label{prop:ratiotest}
    A series $\sum a_n$ of non-zero terms 
    \begin{enumerate}
        \item Converges absolutely if $\limsup \pa{\frac{a_{n+1}}{{a_n}}} < 1$.
        \item Diverges if $\liminf \pa{\frac{a_{n+1}}{{a_n}}} > 1$.
        \item Otherwise the test provides no information.
    \end{enumerate}
\end{proposition}
\begin{proposition}
[Root test]
    \label{prop:roottest}
    Let $\sum a_n$ be a series and $\alpha =\limsup \pa{a_n}^{\frac{1}{n}}$
    The series $\sum a_n$ 
    \begin{enumerate}
        \item Converges absolutely if $\alpha<1$.
        \item Diverges if $\alpha>1$.
        \item Otherwise the test provides no information.
    \end{enumerate}
\end{proposition}
\begin{example}
    [Ratio test]
    Consider $\sum \frac{n^2}{2^n}$, so by \cref{prop:ratiotest},
    \[ \lim\frac{a_{n+1}}{a_n} = \lim\frac{\left(1+\frac{1}{n}\right)^2}{2} = \frac{1}{2}.\]
\end{example}

\begin{example}
    [Using tests]
    Two examples of using both tests,
    \begin{itemize}
        \item Consider $\sum\frac{1}{n!}$. 
        By the ratio test, the ratio is $\frac{1}{n} \rightarrow 0$, so it converges absolutely.
        \item Consider $\sum_{n=0}^{\infty}2^{(-1)^n-n} = 2 + \frac{1}{4}+\frac{1}{2}+\frac{1}{16}+ \frac{1}{8}+\cdots$.
        We see 
        \[ \frac{1}{8} = \liminf \pa{\frac{a_{n+1}}{a_n}} <  \limsup \pa{\frac{a_{n+1}}{a_n}}= 2.\]
        So the ratio test provides no information.
        But the root test yields
        \[\lim(a_n)^{1/n} \lim 2^{\frac{1}{n} - 1}= 2^{-1} = \half. \]
    \end{itemize}
\end{example}
\clearpage
\section{October 4th, 2022}
\subsection{Series test proofs and examples}
We now prove \cref{prop:roottest}.
\begin{proof}
    Suppose $\alpha<1$. Choose $\eps>0$ s.t. $\alpha+\eps < 1$.
    Then $\exists N$ s.t.  
    \[ \alpha-\eps < \sup\ps{\pa{a_n}^{1/n}\mid n > N} < 
    \alpha + \eps. \]
    Hence $\pa{a_n}^{1/n} < \alpha+\eps \forall n > N$.
    $\pa{a_n} < \pp{\alpha+\eps}^n < 1$. Then 
    \[ \sum_{n=N+1}^{k}\pa{a_n} < \sum_{n=N+1}^{k}\pp{\alpha+\eps}^n\]
    is finite, so $\sum a_n$ is also finite.
    
    If $\alpha>1$, then there exists subsequence 
    $\pa{a_n}^{1/n}$ that his limit $>1$,
    hence $\pa{a_n} > 1$ for infinitely many terms and the series diverges.
\end{proof}
Next we prove \cref{prop:ratiotest}.
\begin{proof}
    Since 
    \[ \liminf \pa{\frac{s_{n+1}}{s_n}} \leq \liminf \pa{s_n}^{\frac{1}{n}}
    \leq \limsup \pa{s_n}^{\frac{1}{n}}\leq \limsup \pa{\frac{s_{n+1}}{s_n}}\]
    \[ \implies\liminf \pa{\frac{s_{n+1}}{s_n}} \leq \alpha\leq \limsup \pa{\frac{s_{n+1}}{s_n}},\]
    from which conclusions follow from \cref{prop:roottest}.
\end{proof}

So the ratio test is just a weaker root test, since there are more types of 
functions that satisfy its constraints.

\subsection{Alternating series and integral tests}

\begin{example}
    [Integral test for divergence]
    Consider the sequence $a_n = \frac{1}{n}$. Note that
    \[ \sum_{k=1}^{n}\frac{1}{k} \geq \int_{1}^{n+1}\frac{1}{x}dx = \log\pp{n+1}.\] 
    Taking the limit of everything to $\infty$, 
    $\lim_{n\rightarrow \infty}{\log\pp{n}} = \infty$,
    so series diverges to $\infty$.
\end{example}

\begin{example}
    [Integral test for convergence]
    Consider the sequence $a_n = \frac{1}{n^2}$. Note that
    \[ \sum_{k=1}^{n}\frac{1}{k^2} \leq \int_{1}^{n}\frac{1}{x^2}dx + 1.\]
    Since the integral is finite, the series must converge.
\end{example}

\begin{theorem}
[Alternating series theorem]
    \label{thm:alterseries}
    Suppose you have a sequence $a_i\geq a_{i+1}$ and $a_i\geq 0$ for all $i$,
    and $\lim_{n\rightarrow \infty}{a_n} = 0$. Then 
    \[ \sum_{n=1}^{\infty}(-1)^{n+1}a_n\]
    converges.
\end{theorem}
\begin{proof}
    The subsequence $(s_{2n})$ is increasing, 
    (where $s$ is the series sum) since 
    $s_{2n+2} - s_{2n} = a_{2n+2} + a_{2n+1} \geq 0$.
    Similarly, $(s_{2n-1})$ is decreasing.
    First $s_{2n} \leq s_{2n+1}\forall n$, since $s_{2n+1}-s_{2n}=a_{2n+1} \geq 0$.
    If $m\leq n$, then $s_{2m}\leq s_{2n}\leq s_{2n+1}$. If $m\geq n$,
    $s_{2n+1}\geq s_{2m+1}\geq s_{2m}$
    Hence $(s_{2n})$ and $(s_{2n+1})$ are bounded, and their limits exist.
    Let those limits be $s,t$ respectively.
    \begin{align*}
        t-s &= \lim_{n\rightarrow \infty}{s_{2n+1}} -
        \lim_{n\rightarrow \infty}{s_{2n}} \\ 
        &= \lim_{n\rightarrow \infty}{s_{2nn+1} - s_{2n}} \\
        &= \lim_{n\rightarrow \infty}{a_{2n+1}} = 0.
    \end{align*}
    So $s=t$ and $\lim_{n\rightarrow \infty}{s_n} = s$.
\end{proof}
\subsection{Metric spaces}
Since many of the proofs rely on the triangle inequality, we 
to generalize to more spaces than just $\RR$.

\begin{definition}
    Let $S$ be a set and $d$ be defined for all $(x,y) \in S\times S$
    such that 
    \begin{enumerate}
        \item $d(x,x) = 0 \forall x \in S$ and $d(x,y) > 0 \forall x \neq y,x,y\in S$.
        \item $d(x,y) = d(y,x) \forall x,y\in S$
        \item $d(x,z) \leq d(x,y) + d(y,z) \forall x,y,z\in S$ (\textit{triangle inequality})
    \end{enumerate}
    If all are true, then $d$ is a \vocab{metric}
    and the pair $(S,d)$ is a \vocab{metric space}.
\end{definition}

\begin{example}
    [Euclidean norm]
    For $\RR^k$, let $\vec{x} = (x_1,\cdots, x_k)$, and 
    $\vec{y} = (y_1,\cdots, y_k)$ and define the metric $d$ as 
    $d(\vec{x},\vec{y}) = \sqrt{\sum_{i=1}^{k}(x_i-y_i)^2}$
\end{example}

\begin{definition}
    We extend the convergence property of a sequence $s$ to if 
    \[ \lim_{n\rightarrow \infty}{d(s_n,s)} = 0,\]
    and Cauchy if for each $\eps>0$, $\exists N$ s.t. $m,n > N \implies 
    d(s_n,s_m) < \eps$.
\end{definition}

\begin{definition}
    A metric space is \vocab{complete} if every Cauchy sequence in $S$
    converges to some element in $S$. In $\RR$ the notion of 
    completeness is interchangeable with the completeness axiom.
\end{definition}

\subsection{Topological concepts}
\begin{definition}
    Given a metric space $(S,d)$,
    \begin{enumerate}[a.]
        \item A \vocab{neighborhood} of some element $p\in S$ is a 
        set \[ N_r(p) = \ps{q\mid q \in S, d(p,q)<r}. \] 
        \item A point $p$ is a \vocab{limit point} of a set $E\subseteq S$ 
        if every neighborhood of $p$ contains a $q\neq p$ s.t.
        $q \in E$. If $p\in E$ is not a limit point, it is called an
        \vocab{isolated point}. $E$ is \vocab{closed} if every limit point $p$ of $E$ 
        has $p \in E$.
        \item A point $p$ is an \vocab{interior} point of $E$ if
        there is a neighborhood $N(p)$ s.t. $N\subseteq E$. $E$ is \vocab{open} if every point of $E$ is an 
        interior point of $E$.
        \item The \vocab{complement} of $E$, denoted $E^C$ is the 
        set of all pts. $p \in S$ s.t. $p\notin E$.
        \item $E$ is \vocab{bounded} if there is a real number 
        $M$ and a point $q \in S$ s.t. $d(p,q)< M\forall p \in E$.
    \end{enumerate}
\end{definition}

\begin{definition}
    Two metrics $d_1, d_2$ on $S$ are \vocab{equivalent} if 
    $\forall x \in S$,
    $\forall \eps > 0$, we can find a $\delta >0$ s.t. 
    $N_{\delta}^{(d_1)}(x) \subseteq N_{\eps}^{(d_2)}(x)$ and vice-versa.
\end{definition}

\clearpage
\section{October 6th, 2022}
\subsection{Open and closed sets}

\begin{theorem}
    \label{thm:nbhdopen}
    Every neighborhood is an open set.
\end{theorem}

\begin{proof}
    Consider $E = N_r(p)$, and look at $q \in E$. Then $\exists h > 0$
    s.t. $d(p,q) = r-h$. Consider $s$ s.t. $d(q,s) < h$.
    We know that 
    \[ d(p,s) \leq d(p,q) + d(q,s) < r-h+h = r.\]
    Hence $N_h(q) \subseteq N_r(p) \forall q \in E$, and $E$ is open.
\end{proof}
\begin{proposition}
    If $p$ is a limit point of a set $E$, then every neighborhood of $p$
    contains infinitely many points of $E$.
\end{proposition}
\begin{proof}
    Suppose for contradiction there is a neighborhood $N$ of $p$ that 
    only contains a finite number of points in $E$. Let $q_1,\dots,q_n$ 
    be the points in $N\cap E$ that are distinct.
    Let $r = \min_{m\in\ps{1,\dots,n}} d(p,q_m)$. But that means 
    that $q_i \in N_r(p)$.
\end{proof}
\begin{corollary}
    A finite point set has no limit points.
\end{corollary}
\begin{theorem}
    For a collection of possibly infinite sets $E_\alpha$,
    \[ A = \pp{\bigcup_\alpha E_\alpha}^C = \bigcap_\alpha E_\alpha^C = B.\]    
\end{theorem}
\begin{proof}
    If $x\in A$, then $x \notin \bigcup_\alpha E_\alpha$, so $x \notin E_\alpha$ for all $\alpha$. So $x\in E_\alpha^C$ for all $\alpha$.
    So $x \in \bigcap_\alpha E_\alpha^C = B$.
    Hence $A\subseteq B$, similar proof follows for $B\subseteq A$.
\end{proof}
\begin{theorem}
    $E$ open $\iff$ $E^C$ closed.
\end{theorem}
\begin{proof}
    Suppose $E^C$ is closed. Choose $x \in E$. Since $x\notin E^C$,
    $x$ is not a limit point of $E^C$. Hence $\exists N_r(x)$ s.t.
    $E^C\cap N_r = \varnothing \implies N_r \subseteq E$. Therefore $x$ is 
    an interior point of $E\forall x \in E$, so $E$ is open.

    Suppose $E$ is open. Let $x$ be a limit point of $E^C$. Then 
    every neighborhood contains a point of $E^C$ so $x$ is not an interior 
    point of $E$, so $x$ is not an interior point of $E$. Since $E$ 
    is open, $x\in E^C$ for all limit points of $E$, so $E^C$ is closed. 
\end{proof}
\begin{theorem}
    Results for open and closed sets:
    \begin{enumerate}[a.]
        \item For any collection $\ps{G_\alpha}$ of open sets,
        $\bigcup_\alpha G_\alpha$ is open.
        \item For any collection $\ps{F_\alpha}$ of closed sets,
        $\bigcap_\alpha F_\alpha$ is closed.
        \item For any \textit{finite} collection 
        $G_1,\dots, G_n$ of open sets,
        $\bigcap_{i=1}^n G_i$ is open.
        \item For any \textit{finite} collection 
        $F_1,\dots, F_n$ of closed sets,
        $\bigcup_{i=1}^n F_i$ is closed
    \end{enumerate}
\end{theorem}
\begin{proof}
    (a) Suppose that $\mathcal G = \bigcup_\alpha G_\alpha$. If $x\in \mathcal G$, then $x \in G_\alpha$, and $x$ is an interior point of 
    $G_\alpha$, so $x$ is an interior point of $\mathcal G$. So 
    $\mathcal{G}$ is open.

    (b) Use the fact that $\bigcap_\alpha F_\alpha = \bigcup_\alpha(F_\alpha)^C$.

    (c) Let $H = \bigcap_{i=1}^n G_i$. For any $x\in H$, there exist $N_i$'s with 
    radii $r_i$ s.t. $N_i\subseteq G_i$. Let $r=\min_{i\in\ps{1,\dots,n}}\ps{r_i}$, so $N_r(x)\subseteq G_i$.
    
    (d) Follows similarly to (b).
\end{proof}

\begin{definition}
    If $(X,d)$ is a metric space, and $E\subseteq X$, let $E'$ denote 
    the sets of all limit points of $E$ in $X$. Then then \vocab{closure}
    of $E$ is the set $\ol{E} = E \cup E'$.
\end{definition}

\begin{theorem}
    If $(X,d)$ is a metric space and $E\subseteq X$, then 
    \begin{enumerate}[a.]
        \item $\ol{E}$ is closed.
        \item $E=\ol{E} \iff E$ is closed.
        \item $\ol{E}\subset F$ for every closed set $F\subseteq X$ s.t.   
        $E\subseteq F$.
    \end{enumerate}
\end{theorem}

\begin{proof}
    (a) If $X\ni p\notin \ol{E}$, then $p$ is neither in $E$ nor a limit 
    point of $E$. Hence $p$ has a neighborhood that does not 
    intersect $E$. Therefore $p$ is an interior point of $\ol{E}^C$, so
    $\ol{E}^C$ is open, and $\ol{E}$ is closed.

    (b) If $E = \ol{E}$, then $E$ is closed by (a). If $E$ is closed,
    then $E'\subseteq E$, hence $\ol{E} = E\cup \ol{E} = E$.

    (c) If $F$ is closed and $E \supseteq F$, then $F\supseteq F'$
    and $F'\supseteq E'$. Therefore $F\supseteq \ol{E}$.
\end{proof}

\clearpage
\section{October 11th, 2022}
\subsection{Relative openness}
\begin{proposition}
    Let $E\subseteq \RR$ be nonempty and bounded above.
    Let $y=\sup E$. Then $y\in \ol{E}$, and $y \in E$ if $E$ 
    is closed.
\end{proposition}
\begin{proof}
    $y\in E\implies y\in\ol{E}$, so assume $y\notin E$. For $h>0$,
    there exists $x\in E$ s.t. $y-h<x<y$. Therefore $y$ is 
    a limit point of $E$, and $y\in\ol{E}$.
\end{proof}
Consider $E\subseteq Y \subseteq X$. If $E$ is an open subset of $X$, then 
for each $p\in E$, there exists $r>0$ s.t. 
\[N_r(p) = \ps{q\in X\mid d(p,q)<r}\subseteq E.\]
Note the neighborhood definition depends on the space $X$, and the pair 
$(Y,d)$ is also a metric space. This motivates the following definition.
\begin{definition}
    $E$ is \vocab{open relative to} $Y$ if for each $p\in E$, there 
    exists an $r>0$ s.t.
    \[ N_r^Y(p) =\ps{q\in Y\mid d(p,q)<r}\subseteq E.\]
    In words, for each point $p\in E$, there is a neighborhood 
    on $Y$'s metric that is contained in $E$.
\end{definition}
\begin{example}
    [Relative openness example]
    Suppose $X=\RR$, $Y=[-1,1]$, and $E=(0,1]$. hence $E\subseteq Y\subseteq X$. $E$ is not open relative to $Y$, since $N_{r}^X(1)=(1-r,1+r)$
    is not confined within $E$ for any $r$.
    However, if $E=(0,1)$, $E$ is open relative to $Y$.
\end{example}
\begin{theorem}
    Suppose $Y\subset X$. A subset $E$ is only open relative to $Y$ if 
    and only if $E=Y\cap G$ for some open subset $G$ of $X$. 
\end{theorem}
\begin{proof}
    For each $p\in E$, $\exists r_p>0$ s.t. $N_{r_p}^Y(p) \subseteq E$.
    Let 
    \[ G=\bigcup_{p\in E}N_{r_p}^X(p).\] 
    The union of open sets is open. So $G$ is an open subset of $X$
    and $E\subseteq G\cap Y$. 
\end{proof}
\subsection{Compactness}
\begin{definition}
    An \vocab{open cover} of set $E$ in a metric space is a collection
    $\ps{G_\alpha}$ of open subsets of $X$ such that $E\subseteq \bigcup_\alpha G_\alpha$.
\end{definition}

\begin{definition}
    A subset $K$ of a metric space $X$ is \vocab{compact} if every open cover contains a finite subcover.
\end{definition}

For example, 
if $\ps{G_\alpha}$ is an open cover of $K$, and 
$K$ is compact, then there are finitely 
many indices $\alpha_1,\dots,\alpha_n$ s.t. 
\[ K\subseteq G_{\alpha_1}\cup\dots\cup G_{\alpha_n}. \]

\begin{theorem}
    Suppose $K\subset Y\subset X$. Then $K$ is compact relative to $X
    \iff K$ is compact relative to $Y$.
\end{theorem}

\begin{proof}
    Suppose that $K$ is compact relative to $X$. Let $\ps{V_\alpha}$ 
    be a collection of sets relatively open to $Y$ such 
    that $K \subseteq \bigcup_{\alpha}V_\alpha$. By a previous theorem
    there are sets $G_\alpha$ open relative to $X$ s.t. $V_\alpha=Y\cap G_\alpha$. By compactness,
    \[ K \subseteq G_{\alpha_1}\cup\cdots\cup G_{\alpha_n},\]
    and since $K\subseteq Y$, then 
    \[ K \subseteq V_{\alpha_1}\cup\cdots\cup V_{\alpha_n}. \]

    Conversely, suppose $K$ is compact relative to $Y$,
    and let $\ps{G_\alpha}$ be a collection of open subsets of $X$
    that cover $K$. Define $V_\alpha=Y\cap G_\alpha$.
    Since $V_\alpha$ has a finite cover of $K$, so does $G_\alpha$.
\end{proof}

The conclusion? Compact sets are metric spaces
in their own right. While it does not make sense to talk about 
closed or open metric spaces, 
it does to talk about compact metric spaces.

% Ohmygod he's been doing notes based on the color of his t-shirt

\begin{theorem}
    Compact subsets of metric spaces are closed.
\end{theorem}

\begin{proof}
    Let $K$ be a compact subset of a metric space, and consider 
    $K^C$. Choose $p\in K^C$. For any $q\in K$, let $V_q$ and $W_q$ be 
    neighborhoods of $p$ and $q$ respectively of radius less than 
    $d(p,q)/2$.
    Since $K$ is compact, there are finitely many 
    points $q_1,\dots,q_n$ in $K$ such that $ K\subseteq W_{q_1}\cup \cdots\cup W_{q_n}$.
    If we let $V = V_{q_1}\cap \cdots\cap V_{q_n}$, the $V$ is a neighborhood of $P$ that does not intersect $W$.
    Hence $V\subseteq K^C$, and $p$ is an interior point of $K^C$,
    and $K^C$ is open $\iff$ $K$ is closed.
\end{proof}

\begin{theorem}
    Closed subsets of compact sets are compact.
\end{theorem}

\begin{proof}
    Suppose that $F\subseteq K\subseteq X$, $F$ closed (rel. to $X$),
    and $K$ compact. Let $\ps{V_\alpha}$ be an open cover of $F$.
    If $F^C$ is adjoined to $\ps{V_\alpha}$, we obtain an open cover 
    $\Omega$ of $K$.

    Since $K$ is compact, there is a finite subcollection $\Phi$ of 
    $\Omega$ that covers $K$ and therefore $F$. If $F^C$ is a 
    member of $\Phi$, we can remove it and still retain an 
    open cover of $F$. Thus a finite subcollection of $\ps{V_\alpha}$
    covers $F$, so $F$ is compact.
\end{proof}
\begin{corollary}
    If $F$ is closed, and $K$ is compact, then $F\cap K$ is compact.
\end{corollary}
\clearpage
\section{October 13th, 2022}
\subsection{Compact sets results}
\begin{theorem}
    If $\ps{K_\alpha}$ is a collection of compact subsets of a metric 
    space $X$ s.t. the intersection of every finite subcollection 
    of $\ps{K_\alpha}$ is non-empty, then $\bigcap_\alpha K_\alpha$ is 
    non-empty
\end{theorem}
\begin{proof}
    Fix a member $K_1\in\ps{K_\alpha}$ and let $G_\alpha=K_\alpha^C$.
    Assume that no point of $K_1$ belongs to all $K_\alpha$. Then the 
    set $G_\alpha$ forms an open cover of $K_1$.

    Since $K_1$ is compact, $\exists$ finitely many indices $\alpha_1,\dots,\alpha_n$ s.t. $K_1\subseteq G_{\alpha_1}\cup G_{\alpha_2}\cup \cdots\cup G_{\alpha_n}$.

    But then $K_1\cap K_{\alpha_1}\cap K_{\alpha_2}\cap \cdots\cap K_{\alpha_n}=\varnothing$, so one point of $K_1$ belongs to every set 
    $K_\alpha$, and $\bigcap_\alpha K_\alpha$ is non-empty.
\end{proof}
\begin{corollary}
    \label{co:nonemptyintersection}
    If $\ps{K_n}$ is a sequence of non-empty compact sets such that 
    $K_n\supseteq K_{n+1}\forall n$. Then $\bigcap_{n=1}^{\infty}K_n$ is 
    non-empty.
\end{corollary}

\subsection{Heine-Borel theorem}

\begin{lemma}
    If $E$ is an infinite subset of a compact set $K$, then $E$ has a 
    limit point in $K$.
\end{lemma}

\begin{proof}
    If no point of $K$ were a limit point of $E$, then each $q\in K$
    has a neighborhood $V_q$ such that $\pa{V_q\cap E} \leq 1$.

    Since $E$ has infinitely many points, we know that no finite subcollection of 
    $\ps{V_q}$ can cover $E$,
    and since $E\subseteq K$, then the same is true for $K$,
    implying $K$ is not compact, which is a contradiction.
\end{proof}

\begin{lemma}
    If $\ps{I_k}$ is a sequence of closed intervals in $\RR^1$
    such that $I_n\supseteq I_{n+1}\forall n$. Then $\bigcap_{i=1}^{\infty}I_i$
    is non-empty.
\end{lemma}

\begin{proof}
    Let $I_n$ be the interval $[a_n,b_n]$, and let $E$ 
    be the set of all $a_n$. $E$ is bounded above by $b_1$.
    Let $x=\sup E$. For $m,n \in \NN$,
    \[ a_n\leq a_{m+n}\leq b_{m+n}\leq b_m.\]
    Since $a_m\leq x\leq b_m\forall m \in \NN$, $x\in I_m\ \forall m\in \NN$,
    and $x\in \bigcap_{i=1}^{\infty}$.
\end{proof}

\begin{definition}
    Let $a_i<b_i$ for $i=1,\dots,k$ with $a_i,b_i\in\RR$. Then the set 
    $\vec{x}=(x_1,x_2,\dots,x_k)\in\RR^k$ that satisfy $a_i\leq x_i\leq b_i$
    for all $i$ is called a \vocab{$k$-cell}.
\end{definition}

For notation, we will have all vectors, and their entries, 
written in the way above.
We want to generalize \cref{co:nonemptyintersection}
from nested intervals to nested $k$-cells.

\begin{lemma}
    Let $k>0,k\in\NN$. If $\ps{I_n}$ is a sequence of $k$-cells s.t.
    $I_n\supseteq I_{n+1}\forall n$, then $\bigcap_{n=1}^{\infty}I_n$
    is non-empty.
\end{lemma}
\begin{proof}
    Let $I_n$ be the set of all points $\vec{x}\in \RR^k$
    s.t. $a_{n,j}\leq x_k\leq b_{n,j}$ ($1\leq j\leq k,n\in\NN$),
    and put $I_{n,j}=[a_{n,j},b_{n,j}]$. For each dimension, 
    there exists $x_j^*$ s.t. $a_{n,j}\leq x_j^*\leq b_{n_j}$ 
    for all $j$. So $\vec{x}^*=(x_1^*,x_2^*,\dots,x_k^*)\in \bigcap_{n=1}^{\infty}I_n$ 
\end{proof}
\begin{lemma}
    Every $k$-cell is compact.
\end{lemma}
\begin{proof}
    Let $I$ be a $k$-cell consisting of points $\vec{x}$
    where $a_j\leq x_j\leq b_j$ for all $j$, and let 
    \[ \delta = \sqrt{\sum_{j=1}^{k}(b_j-a_j)^2}.\]
    Then $d(\vec{x},\vec{y})\leq \delta$ for $\vec{x},\vec{y}\in I$.

    Suppose there exists an open cover $\ps{G_\alpha}$ of $I$ 
    that has no finite subcover. Let $c_k=\frac{a_j+b_j}{2}$.
    Then the intervals $[a_j,c_j]$ and $[c_j,b_j]$ determine 
    $2^k$ $k$-cells $\ps{Q_i}$, whose union is $I$.

    At least one of the $Q_i$'s cannot be covered by any finite 
    subcollection of $\ps{G_\alpha}$, let that be $I_1$. Now 
    subdivide $I_1$ and continue this process to get a 
    sequence $\ps{I_j}_{j\in\NN}$ s.t.
    \begin{enumerate}
        \item $I\supseteq I_1\supseteq I_2\supseteq\cdots$
        \item $I_n$ is not covered by any finite subcollection of $\ps{G_\alpha}$
        \item If $\vec{x},\vec{y}\in I_n$, then 
        \[d(\vec{x},\vec{y})\leq \frac{\delta}{2^n}. \]
    \end{enumerate}
    The previous result tells us that there is some $\vec{x}^*$ 
    s.t. $\vec{x}^*\in I_n\forall n$. We must have $\vec{x}^*\in G_\alpha$
    for some $\alpha$. Since $G_\alpha$ is open, $\exists r>0$ s.t.
    the $N_r(\vec{x}^*)\subseteq G_\alpha$.

    By choosing a suitably large $n$ s.t. $2^{-n}\delta<r$. Then 
    $I_n\subseteq G_\alpha$. So $G_\alpha$ covers $I_n$ contradicting 
    our assumption.
\end{proof}
\begin{theorem}
[Heine-Borel theorem]
    \label{thm:heineborelthm}
    If $E\subseteq \RR^k$, then the following statements are equivalent:
    \begin{enumerate}[(a)]
        \item $E$ is closed and bounded.
        \item $E$ is compact 
        \item Every infinite subset of $E$ has a limit point in $E$.
    \end{enumerate}
\end{theorem}
\begin{proof}
    If (a) holds, then we can enclose $E$ in a $k$-cell $I$.
    Since closed subset of compact sets are compact, $E$ is compact,
    so (a) $\implies$ (b).

    (b) $\implies$ (c) is finished by a previous theorem.

    To show (c) $\implies$ (a), suppose that $E$ is not bounded.
    Then $E$ contains points $\vec{x}_n$ with 
    $\pa{\vec{x}_n}>n\forall n \in \NN$. The set consisting of 
    these points is infinite, but has no limit point in $\RR^k$.
    Hence (c) implies that $E$ is bounded. 
    
    If $E$ is not closed, 
    then there is a point $\vec{x}_0\in \RR^k$ that is a limit point of $E$,
    but is not contained in $E$. Therefore $\exists \vec{x}_n\in E$ s.t.
    $\pa{\vec{x}_n-\vec{x}_0}<\frac{1}{n}\forall n \in \NN$.
    Let $S$ be the set of all such $x_n$.
    $S$ is infinite and has a limit point $\vec{x}_0$, and no other limit points in $\RR^k$. Choose $\vec{y}\in\RR^k$, where $\vec{y}\neq \vec{x}_0$.
    Then 
    \[ d(\vec{x}_n,\vec{y})\geq d(\vec{x}_0,\vec{y})-d(\vec{x}_n,\vec{y})
    \geq d(\vec{x}_0,\vec{y})-\frac{1}{n}\geq \frac{1}{2}d(\vec{x}_0,\vec{y}).\]
    For all but finitely many $n$. Thus $\vec{y}$ is is not a limit point of 
    $S$, and $E$ is closed.
\end{proof}
\subsection{Functions and continuity}
\begin{definition}
    The \vocab{domain} of a function $f$ is the set on which $f$ is defined,
    and is denoted $\dom(f)$.
\end{definition}
For a \textit{real valued function}, $\dom(f)\subseteq \RR$, and $f(x)\in\RR$ for 
all $x\in\dom(f)$. Sometimes the domain is omitted, and we 
assume that the function is valid on the \textit{natural domain},
the set of all points in $\RR$ on which the function is well-defined.
\begin{definition}
    Let $f$ be a real valued function whose domain is a subset of $\RR$.
    The function $f$ is \vocab{continuous at $x_0$} in $\dom(f)$ if 
    for every sequence $(x_n)$ in $\dom(f)$ that converges 
    to $x_0$,  
    \[\lim_{n\rightarrow \infty}f(x_n)=f(x_0). \]
    If $f$ is continuous at each point of a set $S\subseteq \dom(f)$, 
    then $f$ is said to be \vocab{continuous on $S$}.
\end{definition}
\clearpage
\section{October 18th, 2022}
Prof. Rycroft was out of town today, 
so the lecture notes are from a virtual meeting.
Lecture recording will be in Kaltura.
\subsection{Continuing continuity}
\begin{theorem}
    \label{thm:continuity}
    Let $f : S \rightarrow \RR$, where $S\subseteq \RR$. Then $f$ 
    is continuous at $x_0\in S$ if and only if $\forall\eps>0$,
    $\exists\delta>0$ s.t. $x\in S$ and $\pa{x-x_0}<\delta$, then 
    $\pa{f(x)-f(x_0)}<\eps$.
\end{theorem}

\begin{proof}
    Suppose that $\forall\eps>0$,
    $\exists\delta>0$ s.t. $x\in S$ and $\pa{x-x_0}<\delta$, then 
    $\pa{f(x)-f(x_0)}<\eps$ is true.
    Take the sequence $(x_n)$ in $\dom(f)$ s.t. $\lim_{n\rightarrow \infty}x_n = x$.
    Let $\eps>0$. $\exists \delta>0$ s.t. $x\in S$ and $\pa{x-x_0} < \delta
    \implies \pa{f(x)-f(x_0)}<\eps$. We know $\exists N$ s.t. 
    $n>N\implies \pa{x_n-x_0} < \delta$, and therefore 
    $\pa{f(x_n)-f(x_0)} < \eps$.

    Assume that $f$ is continuous but the theorem fails.
    So for each $n\in \NN$, for $\pa{x_n-x_0} < \frac{1}{n}$.
    $\exists x_n\in\dom(f)$ s.t. $\pa{f(x_n)-f(x_0)}\geq \eps$.
    But then $\lim f(x_n)\neq f(x_0)$.
\end{proof}

\begin{example}
    [Continuity with rapid oscillations]
    Consider \[ f_1(x) = \begin{cases}
        x\sin\frac{1}{x} &x\neq 0, \\
        0 &x=0.
    \end{cases}\]
    Note that $\pa{f_1(x) -f_1(0)} = \pa{x\sin\frac{1}{x}}<\pa{x}$
    if $x\neq 0$ and $0$ otherwise.
    Let $\delta = \eps$ in \cref{thm:continuity}, and finish to show that
    $f_1$ is continuous at $0$.
\end{example}

\begin{proposition}
    Let $f$ be a real valued function with $\dom(f)\subseteq \RR$.
    If $f$ is continuous at $x_0$ in $\dom(f)$, then $\pa{f}$ and 
    $kf$ are too.
\end{proposition}

\begin{proposition}
    Let $f$ and $g$ be real valued function continuous at $x_0\in\RR$.
    Then 
    \begin{enumerate}
        \item $f+g$ is continuous at $x_0$.
        \item $fg$ is continuous at $x_0$.
        \item $f/g$ is continuous at $x_0$ provided $g(x_0)\neq 0$.
    \end{enumerate}
\end{proposition}
\begin{proof}
    First two follow from sequence convergence theorems.
    For (3), examine $(x_n)$ in $\dom(f)\cap \ps{x\in\dom(g)\mid g(x)\neq 0}$ s.t. $x_n\to x_0$. Then 
    \[ \lim_{n\rightarrow \infty}\pp{\frac{f}{g}}\pp{x_n} = \lim_{n\rightarrow \infty}\frac{f(x_n)}{g(x_n)}=\frac{f(x_0)}{g(x_0)}.
    \qedhere \]
\end{proof}
\begin{proposition}
    The composition of continuous function is continuous.
\end{proposition}

\begin{example}
    The "max" function is continuous. Let $\max(f,g) = \frac{1}{2}(f+g) + \frac{1}{2}\pa{f-g}$. We have shown that all the operations
    performed of $f$ and $g$ preserve continuity.
\end{example}

\begin{theorem}
    Let $f$ be a continuous real-valued function on a closed interval 
    $[a,b]$. Then $f$ is bounded and achieves its max/min values.
\end{theorem}
\begin{proof}
    Suppose that $f$ is not bounded. Then for each $n\in \NN$, 
    $\exists x_n$ s.t. $\pa{f(x_n)}>n$. By \cref{thm:bwt},
    there is a subsequence $(x_{n_k})$ that converges to $x_0$.
    $x_0$ must be within the closed interval. Since $f$ is continuous at 
    $x_0$, $\lim_{k\rightarrow \infty}f(x_{n_k}) =f(x_0)$.
    But $\lim_{k\rightarrow \infty}\pa{f(x_{n_k})} = \infty$,
    which is a contradiction, so $f$ is bounded.
    
    Let $M = \sup\ps{f(x)\mid x\in[a,b]}$. $M$ is finite.
    Then for each $n\in \NN$, $\exists y_n\in[a,b]$ s.t. $M-\frac{1}{n}<f(y_n) < M$. Then $\lim_{n\rightarrow \infty}f(y_n) = M$.
    By \cref{thm:bwt}, $\exists$ a convergent subsequence $y_{n_k}$
    that has limit $y_0$ in $[a,b]$. 
    Since $f$ is continuous at $y_0$, then 
    \[ f(y_0) = \lim_{k\rightarrow \infty}f(y_{n_k}) = \lim_{n\rightarrow \infty}f(y_n) = M. \]
    The same argument follows for the minimum.
\end{proof}
\begin{theorem}
[Intermediate Value Theorem]
    \label{thm:intermediatevalthm}
    Let $f$ be a continuous real-valued function on an interval $I$.
    Then whenever $a,b\in I$, $a<b$, and $f(a) < y < f(b)$ or 
    $f(b) < y < f(a)$, then $\exists$ at least $1$ $x\in(a,b)$ s.t.
    $f(x) = y$.
\end{theorem}
\begin{proof}
    Let $S = \ps{x\in [a,b] \mid f(x)<y}$.
    $a\in S$ so $S$ is non-empty, so $\sup S = x_0$, where $x_0 \in [a,b]$.
    For all $n\in \NN$, $x_0-\frac{1}{n}$ is not an upper bound for $S$.
    So $\exists s_n\in S$ s.t. $x_0-\frac{1}{n}<s_n\leq x_0$. Hence 
    $\lim_{n\rightarrow \infty}s_n=x_0$, and since $f(s_n)<y \forall n$,
    \[ f(x_0) = \lim_{n\rightarrow \infty}f(s_n)\leq y.\] 

    Now let $t_n = \min\ps{b, x_0+\frac{1}{n}}$. 
    Since $x_0\leq t_n\leq x_0+\frac{1}{n}$, then 
    $\lim t_n = x_0$.
    $t_n\in[a,b]$, but $t_n\notin S$ so $f(t_n)\geq y$, and 
    \[ f(x_0) = \lim_{n\rightarrow \infty}f(t_n)\geq y.\]

    Using both results, $f(x_0) = y$.
\end{proof}
\begin{corollary}
    If $f$ is a continuous real-valued function on an interval $I$, 
    then $f(I)$ is also an interval or a single point.
\end{corollary}
\begin{proof}
    Given $y_0,y_1\in f(I)$, then \cref{thm:intermediatevalthm} tells us 
    that $y_0<y<y_1$ for $y \in f(I)$.
\end{proof}
\clearpage
\section{October 20th, 2022}
Zoom lecture again\dots
\subsection{Strictly increasing functions}
\begin{definition}
    Consider $f(x)$ on the interval $I$. $f(x)$ is \vocab{strictly 
    increasing} if for all $x,y\in I$, $x<y \implies f(x)<f(y)$.
    In cases like this, an \vocab{inverse function} $f\inv$ can 
    be unambiguously defined so that $(f\inv\circ f)(x) = x$.
\end{definition}

\begin{theorem}
    Let $g$ be a strictly increasing function on an interval $J$ 
    such that $g(J)$ is an interval $I$. Then $g$ is continuous on $J$.
\end{theorem}

\begin{proof}
    Consider $x_0\in J$ so that it is not 
    an endpoint. Hence $g(x_0)$ is not an endpoint, and $\exists \eps_0$
    s.t. \[ (g(x_0)-\eps_0, g(x_0) +\eps_0) \subseteq I.\]
    Assume $\eps<\eps_0$. $\exists x_1,x_2\in J$ s,t,
    \[
        g(x_1)=g(x_0) - \eps, \quad g(x_2)=g(x_0) + \eps.
    \]
    Then $x_1<x_0<x_2$ since it is strictly 
    increasing. Similarly for $x\in(x_1,x_2)$, $g(x_1) < g(x)  < g(x_2)$,
    then $\pa{g(x_0)-g(x)}<\eps$. Let $\delta = \min\ps{x_2-x_0, x_0-x_1}$.
    Then $\pa{x-x_0}<\delta \implies \pa{g(x)-g(x_0)}<\eps$.
\end{proof}

\begin{theorem}
    Let $f$ be a continuous, strictly increasing function on the 
    interval $I$. $J=f(I)$ is an interval by previous result,
    and $f\inv$ represent a function on $f(I)$ that is continuous 
    and strictly increasing.
\end{theorem}

\begin{proof}
    Continuity is given by the previous theorem.

    Suppose $a,b\in J$ s.t. $a<b$. Then $\exists c,d \in I$ s.t. 
    $f(c)=a$ and $f(d)=b$. $a\neq b\implies c\neq d$, so 
    $c<d$ (if $c>d$, then $f(c)>f(d)$). Hence, since $c=f\inv(a)$
    and $d=f\inv(b)$ we see $f\inv(a)<f\inv(b)$.
    Therefore $f\inv$ is strictly increasing.
\end{proof}

We can think of this a partial converse to \cref{thm:intermediatevalthm}:
A strictly increasing function with the intermediate value property 
is continuous.
\subsection{Uniform continuity}
Recall that $f$ is continuous on $S\subseteq\dom(f) \implies$
\[ \forall x_0\in S, \forall \eps>0, \exists \delta>0 \text{ s.t. } \]
\[ x\in\dom(f) \text{ and } \pa{x-x_0}<\delta\implies \pa{f(x)-f(x_0)}<\eps. \]
Note that the voice of $\delta$ depends on the value of $\eps$ and $x_0$.
For example, $f(x) = x\inv$ on $(0,\infty)$ has vastly different $\delta$
choices for different $\eps$ and $x_0$.

\begin{example}
Consider showing $f(x)=x\inv$ is continuous at a point $x_0>0$.
Then 
\[ f(x) - f(x_0) = x\inv - x_0\inv = \frac{x_0-x}{xx_0}.\]
Pick $\eps>0$. Suppose that $\pa{x-x_0} < x_0/2$, then $\frac{x_0}{2}<x<\frac{3x_0}{2}$. Then 
\[\pa{f(x) - f(x_0)} =\frac{\pa{x_0-x}}{xx_0} < \frac{\pa{x_0-x}}{\frac{x_0}{2}x_0} = \frac{2\pa{x_0-x}}{x_0^2}. \] 
Now suppose that $\delta=\min\ps{\frac{x_0}{2}, \frac{\eps x_0^2}{2}}$. 
Then $\pa{x_0-x} < \delta$ implies 
\[\pa{f(x)-f(x_0)} < \frac{2}{x_0^2}\cdot\frac{\eps x_0^2}{2}= \eps. \]
\end{example}
Hence $f$ in the example is continuous at $x_0$, but $\delta$ gets small as 
$x_0$ gets small due to the steepness of $\frac{1}{x}$.
This motivates a definiton, 

\begin{definition}
    Let $f$ be a real-valued function on $S\subseteq \RR$. Then $f$ is 
    \vocab{uniformly continuous} on $S$ if 
    \[ \forall \eps>0, \exists \delta>0 \text{ s.t. }\]
    \[ \forall x,y\in S \text{ and } \pa{x-y} < \delta \implies \pa{f(x)-f(y)}< \eps.\]
\end{definition}

\begin{example}
    We consider $f(x) = x^{-1}$ again. It is uniformly continuous 
    on the interval $[a,\infty)$ for $a>0$. Let $\eps>0$ and consider any $x,y\geq a$. Pick $\delta=\min\ps{\frac{a}{2}, \frac{\eps a^2}{2}}$.
    If $\pa{x-y}<\delta$, then $\pa{f(x)-f(y)} <\delta$ from before.

    However $f(x)$ is not uniformly continuous on $(0,\infty)$.
    To show this, we will prove that 
    $\forall \delta>0$, $\exists x,y \in (0,\infty)$
    s.t. $\pa{x-y} < \delta$, and yet $\pa{f(x)-f(y)}\geq 1$.
    If $\delta>1$, choose $x=1, y=\frac{1}{2}$.
    Otherwise, if $\delta\leq 1$, choose $x=\delta, y=\frac{\delta}{2}$,
    so $\pa{\frac{1}{\delta}-\frac{1}{\delta/2}} = \frac{1}{\delta}\geq 1$.
\end{example}

\begin{theorem}
    [Continuous on closed interval $\implies$ uniformly continuous]
    If $f$ is continuous on $[a,b]$, then $f$ is uniformly continuous 
    on $[a,b]$.
\end{theorem}

\begin{proof}
    Assume $f$ is not uniformly continuous on $[a,b]$. 
    Then $\exists \eps>0$ s.t. $\forall \delta>0, \exists x,y \in [a,b]$
    s.t. $\pa{x-y} < \delta$ but $\pa{f(x)-f(y)}\geq \eps$.

    If this is true, then define sequences $x_n, y_n\in [a,b]$ s.t.
    $\pa{x_n-y_n}<\frac{1}{n}$, but $\pa{f(x_n)-f(y_n)} \geq \eps$.
    By \cref{thm:bwt}, there is a subsequence $(x_{n_k})$ 
    which converges. But if $x_0=\lim_{k\rightarrow \infty}x_{n_k}$,
    then $x_0\in [a,b]$. In addition, $\lim_{k\rightarrow \infty}y_{n_k}=x_0$ as well. But since $f$ is continuous at $x_0$, 
    \[ f(x_0) = \lim_{k\rightarrow \infty}f(x_{n_k})= \lim_{k\rightarrow \infty}f(y_{n_k}) \implies \lim_{k\rightarrow \infty}f(x_{n_k})-f(y_{n_k}) = 0, \]
    which contradicts the asumption that 
    $\pa{f(x_{n_k})-f(y_{n_k})} \geq \eps$
\end{proof}

\begin{theorem}
    [Uniformly continuous preserves Cauchy]
    If $f$ is uniformly continuous on a set $S$ and $(s_n)$ is Cauchy 
    in $S$, then $(f(s_n))$ is Cauchy.
\end{theorem}

\begin{proof}
    For $\eps>0$, $\exists \delta$ s.t. $x,y\in S$ $\pa{x-y}<\delta 
    \implies \pa{f(x)-f(y)}<\eps$. Then since $(s_n)$ is Cauchy, 
    $\exists N$ s.t. $n,m > N \implies \pa{s_n-s_m}<\delta\implies \pa{f(s_n)-f(s_m)}<\eps$.
    This result requires uniform continuity.
\end{proof}
\subsection{Introducing limits of functions}
\begin{definition}
    Let $S\subseteq \RR$, and $a\in \RR\cup\ps{\pm\infty}$ that 
    is the limit of some sequence in $S$. Let $L\in\RR\cup\ps{\pm\infty}$.
    We write 
    \[ \lim_{x\rightarrow a^S}f(x) = L\]
    given that
    \begin{itemize}
        \item $f$ is a function defined on $S$.
        \item For every sequence $(x_n)$ in $S$ with limit $a$, we 
        have $\lim_{n\rightarrow \infty}f(x_n) = L$.
    \end{itemize}
\end{definition}
We can conclude the limit exists if and only if $f$ is continuous $a$
on $S$. 
\begin{definition}
    Here are some standard definitions:
    \begin{enumerate}[a.]
        \item For $a\in\RR$, write $\lim_{x\rightarrow a}f(x)=L$ if 
        $\lim_{x\rightarrow a^S}f(x)=L$ for some $S=J\setminus\ps{a}$
        where $J$ is an open interval containing $a$.
        \item \vocab{Positive and negative limits} are defined as 
        \[\lim_{x\rightarrow a^+}f(x)=L \]
        if $\lim_{x\rightarrow a^S}f(x)=L$ for some open interval $S=(a,b)$,
        or 
        \[\lim_{x\rightarrow a^-}f(x) = L \]
        if $\lim_{x\rightarrow a^S}f(x)=L$ for some open interval $S=(c,a)$.
    \end{enumerate}
\end{definition}
\clearpage
\section{October 25th, 2022}
\subsection{Limits of functions}

\begin{definition}
    \vocab{Infinite function limits} are written 
    \[ \lim_{x\rightarrow \infty}f(x) = L \]
    if $\lim_{n\rightarrow \infty^S}f(x)=L$, where $S=(c,\infty)$.
\end{definition}

\begin{proposition}
    [Function limit properties]
    Let $f_1,f_2$ be function such that 
    \[L_1=\lim_{x\rightarrow a^S}f_1(x) \qquad\text{and}\qquad  
    L_2=\lim_{x\rightarrow a^S}f_2(x).\]
    \begin{enumerate}
        \item $\lim_{x\rightarrow a^S}(f_1+f_2)(x) = L_1+L_2$
        \item $\lim_{x\rightarrow a^S}(f_1f_2)(x) = L_1L_2$
        \item $\lim_{x\rightarrow a^S}(f_1+f_2)(x) = L_1/L_2$ provided 
        $f(x)\neq0 \forall x\in S$ and $L_2\neq 0$.
    \end{enumerate}
\end{proposition}

\begin{proof}
    (1) Consider $x_n$ in $S$ with limit $a$. 
    Then 
    \[
        \lim_{n\rightarrow \infty}(f_1+f_2)(x_n)=\lim_{n\rightarrow \infty}f_1(x_n)+\lim_{n\rightarrow \infty}f_2(x_n)=L_1+L_2.
    \]
    Similarly, (2) and (3) are true.
\end{proof}

\begin{theorem}
    Let $f$ be a function defined on $S\supseteq\RR$, and $a\in\RR$
    is a limit of some seq. in $S$. Then 
    \[ \lim_{x\rightarrow a^S}f(x)=L\in \RR\]
    if and only if 
    \[\forall \eps>0, \exists\delta>0 \text{ s.t. } x\in S\text{ and} \]
    \[\pa{x-a}<\delta \text{ then } \pa{f(x)-L}<\eps. \]
\end{theorem}

\begin{proof}
    Consider a sequence in $S$ s.t. $\lim_{n\rightarrow \infty}x_n=a$.
    Goal: show $\lim_{n\rightarrow \infty}f(x_n)=L$.

    Assume the second part is true. Then $\exists N$ s.t. 
    $n>N\implies \pa{x_n-a}<\delta\implies\pa{f(x)-L}<\eps.$
    Hence $\lim_{n\rightarrow \infty}f(x_n)=L$.

    Now assume that $\lim_{x\rightarrow a^S}f(x)=L$, but the second part 
    fails. Then $\exists \eps>0$ s.t. $\forall\delta>0,x\in S$ and 
    $\pa{x-a}<\delta$ does not imply $\pa{f(x)-L}<\eps$.
    For each $n\in\NN, \exists x_n\in S$ where $\pa{x_n-a}<\frac{1}{n}$
    while $\pa{f(x_n)-L}\geq\eps$. So $x_n\to a$, but 
    $\lim_{n\rightarrow \infty}f(x_n)=L$ fails, which is a contradiction.
\end{proof}
Alternatively,
let $f$ be defined on $I\setminus\ps{a}$, where $I$ is an open interval 
and $a\in I$. $\lim_{x\rightarrow a}f(x)=L\in\RR$ if and only if 
\[\forall \eps>0, \exists\delta>0 \text{ s.t. } 0<\pa{x-a}<\delta 
\implies \pa{f(x)-L}<\eps. \]

\begin{theorem}
    Let $f$ be a function defined on $I\setminus\ps{a}$. 
    $\lim_{x\rightarrow a}f(x)$ exists $\iff \lim_{x\rightarrow a^+}f(x)$
    and $\lim_{x\rightarrow a^-}f(x)$ exist and are equal, in 
    which case all mentioned limits are equal.
\end{theorem}
\begin{proof}
    If $\lim_{x\rightarrow a}f(x)=L$, then 
    \[\forall\eps>0, \exists\delta>0 \text{ s.t. } 0<\pa{x-a}<\delta\implies \pa{f(x)-L}<\eps.\]
    It follows that 
    \[ \forall\eps>0, \exists\delta>0 \text{ s.t. } a<x<a+\delta\implies \pa{f(x)-L}<\eps\]
    \[ \forall\eps>0, \exists\delta>0 \text{ s.t. } a-\delta<x<a\implies \pa{f(x)-L}<\eps\]
    Therefore both limits are equal.

    Conversely, choose $\eps>0$. Then 
    \[ \forall\eps>0, \exists\delta_1>0 \text{ s.t. } a<x<a+\delta_1\implies \pa{f(x)-L}<\eps\]
    \[ \forall\eps>0, \exists\delta_2>0 \text{ s.t. } a-\delta_2<x<a\implies \pa{f(x)-L}<\eps\]
    Let $\delta=\min\ps{\delta_1,\delta_2}$. Then 
    \[ 0<\pa{x-a}<\delta\implies \pa{f(x)-L}<\eps\implies \lim_{x\rightarrow a}f(x)=L.\qedhere\]
\end{proof}
\subsection{Power series}
\begin{definition}
    Let $(a_n)$ be a real number seq. Then 
    \[\sum_{n=0}^{\infty}a_nx^n \]    
    is called a \vocab{power series}. 
    We use the convention in power series that $0^0=1$.
\end{definition}
We can use power series to approximate other functions. For example,
\[\sin x = x-\frac{x^3}{3!}+\frac{x^5}{5!}-\cdots.\]

\begin{theorem}
    For the power series $\sum a_nx^n$, let $\beta=\limsup\pa{a_n}^{1/n}$,
    and $R=\beta\inv$ (let $R=\infty$ if $\beta=0$ and vice-versa).
    
    Then the power series converges for $\pa{x}<R$, and diverges for 
    $\pa{x}>R$. $R$ is called the \vocab{radius of convergence}.
\end{theorem}

\begin{proof}
    Using the root test, for $x$,
    let $\alpha_x=\limsup\pa{a_nx^n}^{1/n}=\limsup\pa{a_n}^{1/n}\pa{x}
    =\pa{x}\limsup\pa{a_n}^{1/n}=\pa{x}\beta$.

    If $0<\beta<\infty$, then $\alpha_x=\beta\pa{x}=\frac{\pa{x}}{R}$.
    Then if $\pa{x}<R$, then $\alpha_x<1$, and the series converges, and 
    if $\pa{x}>R$, then $\alpha_x>1$, and the series diverges.
\end{proof}

\begin{corollary}
    [Power series convergence properties]
    For a power series $\sum a_nx^n$, either
    \begin{enumerate}
        \item It converges $\forall x\in\RR$.
        \item It converges at $x=0$ only.
        \item It converges $x\in I$, where $I$ is an interval, but not necessarily open or closed.
    \end{enumerate}
\end{corollary}

\begin{example}
    [Endpoints are not guaranteed]
    \[ \sum_{n=0}^{\infty}x^n,\qquad \sum_{n=0}^{\infty}n\inv x^n,\qquad \sum_{n=0}^{\infty}n^{-2} x^n\]
    all have $R=1$, but the first doesn't converge for $x=\pm 1$,
    the second converges only for $x=-1$, and the last converges for 
    $x=\pm 1$.
\end{example}
We can write more generally for any point $x_0\in \RR$,
\[ \sum_{n=0}^{\infty}a_n(x-x_0)^n.\]
Any partial sum will be continuous (and differentiable),
but this doesn't guarantee that the entire power series will also 
be continuous.

% For example,
% \[ f(x)=\begin{cases}
% 1 &\text{if }x=0\\
% 1 &\text{if }x\neq0
% \end{cases}\]
% and let $f_n(x)=\min(1-n\pa{x},0)$.
% This motivates a stronger definition of convergence: 
% \textit{uniform convergence}. 
% Power series do satisfy this stronger definition.
\begin{definition}
    Let $(f_n)$ be a sequence of real-valued functions defined on 
    $S\subseteq\RR$. Then the sequence \vocab{converges pointwise}
    to a function $f$ on $S$ if
    \[\lim_{n\rightarrow \infty}f_n(x)=f(x)\forall x\in S.\]
    Then write that 
    \[\lim_{n\rightarrow \infty}f_n=f \qquad f_n\to f,\]
    and both are pointwise.

    In terms of the $\eps$-$\delta$ definiton, 
    \[ \forall x\in S, \forall \eps>0, \exists N\text{ s.t. } 
    n>N \implies \pa{f_n(x)-f(x)}<\eps.\]
\end{definition}
\clearpage
\section{October 27th, 2022}
\subsection{Uniform convergence of functions}
\begin{definition}
    Let $(f_n)$ be a seq. of real-valued functions defined on 
    $S \subseteq \RR$. Then $f_n$ \vocab{converges uniformly} on $S$
    to $f$ if 
    \[
        \forall \eps>0, \exists N \text{ s.t. } \pa{f_n(x)-f(x)}<\eps \
        (\forall x\in S, n > N)
    \]
    In this case, write $\lim_{n\rightarrow \infty}f_n=f$ uniformly.
\end{definition}
For any $\eps>0$, the $f_n$ have to eventually lie within a strip 
of width $\eps$ around $f$.

\begin{example}
    Consider $f_n=(1-\pa{x})^n$. Then $\pa{f_n(x)-f(x)}$ should 
    eventually be smaller than $\eps=\frac{1}{2}$. However, 
    the function will always pass the strip $f(x)+\eps<f_n(x)$.

    Let $x=1-2^{-\frac{1}{N+1}}$. Then 
    \[
        (1-x)^{N+1}=\frac{1}{2}.
    \]
    Therefore $\pa{f_{N+1}(x)-f(x)}=\frac{1}{2}$.
\end{example}
Uniformly convergent series of functions are a subset of pointwise convergent series
of functions. 

\begin{example}
    [Uniform convergence with rapid oscillations]
    For example, 
    \[
        f_n(x)=\frac{1}{n}\sin n^2x,
    \]
    which forms lower amplitude sine waves with higher frequency
    as $x\to 0$.
    For $\eps>0$, $\exists N$ s.t. $n>N$
    $\pa{f_n(x)-0}\leq \frac{1}{n}<\frac{1}{N}=\eps$.
\end{example}

\begin{theorem}
    Let $(f_n)$ be a series of functions $S \subseteq\RR$ and suppose 
    $f_n\to f$ uniformly on $S$ and $\dom(f)=S$.
    If each $f_n$ is continuous at $x_0\in S$, then $f$ is continuous
    at $x_0$.
\end{theorem}

\begin{proof}
    Let $\eps>0$. $\exists N\in\NN$ s.t. $n>N\implies \pa{f_n(x)-f(x)}<\frac{\eps}{3}\forall x\in S$, so $\pa{f_{N+1}(x)-f(x)}<\frac{\eps}{3}\forall x\in S$. $f_{N+1}$ is continuous at $x_0$, so 
    $\exists\delta>0$ s.t. $x\in S$ and $\pa{x-x_0}<\delta\implies
    \pa{f_{N+1}(x)-f_{N+1}(x_0)}<\frac{\eps}{3}$.

    Then $x\in S$ and $\pa{x-x_0}<\delta$ implies 
    \begin{align*}
        \pa{f(x)-f(x_0)}&\leq \pa{f(x)-f_{N+1}(x)} + \pa{f_{N+1}(x)-f_{N+1}(x_0)}  + \pa{f_{N+1}(x_0)-f(x_0)} \\
        &< \eps.\qedhere
    \end{align*}
\end{proof}

\begin{definition}
    A seq. $(f_n)$ of functions defined on $S \subseteq\RR$ is 
    \vocab{uniformly Cauchy} at $x$ if 
    \[
        \forall \eps>0 \exists N \text{ s.t. } \pa{f_n(x)-f_m(x)}<\eps \ 
        (\forall x\in S, m,n>N).
    \]
\end{definition}

\begin{theorem}
    [Uniformly Cauchy $\implies$ uniformly convergent]
    Let $(f_n)$ be series of functions that are uniformly Cauchy on 
    $S \subseteq\RR$. Then $\exists f$ on $S$ s.t. $f_n\to f$ uniformly.
\end{theorem}
\begin{proof}
    Choose $\eps>0$. Then for fixed $x_0\in S$, 
    $\pa{f_n(x_0)-f_m(x_0)} <\eps\ \forall m,n>N$.
    Hence $f_n(x_0)$ is a Cauchy seq. so it must converge to $f(x_0)$.

    Since this applies to any $x_0\in S$, $f_n\to f$ pointwise.
    To show convergence is uniform, choose $\eps>0$. Then $\exists N$
    s.t. $\pa{f_n(x)-f_m(x)}<\eps/2$. Fix $m>N$. 
    Then $\forall n>N$, 
    \[
        f_n(x)\in\pp{f_m(x)-\frac{\eps}{2}, f_m(x)+\frac{\eps}{2}}.
    \]
    Then 
    \[
        \lim_{x\rightarrow a}f_m(x) = f(x) \in \left[f_m(x)-\frac{\eps}{2}, f_m(x)+\frac{\eps}{2}\right]\implies \pa{f(x)-f_m(x)}\leq \frac{\eps}{2}<\eps.\qedhere
    \]
\end{proof}
\subsection{Application to power series}
\begin{proposition}
    Consider a sequence of partial sums of function $(\sum_{k=0}^{n}g_k)$
    defined on $S \subseteq\RR$. If each $g_k$ continuous on $S$
    and $g_k\to g$ uniformly and $g$ continuous on $S$,
    then $\sum_{k=0}^{\infty}g_k$ is continuous.
\end{proposition}

\begin{proof}
    Let $f_n=\sum_{k=0}^{n}g_k$. 
    $f_n$ is continuous, and $f_n\to f$ uniformly, and $f$ is also continuous.
\end{proof}

\begin{proposition}
    [Analog of Cauchy criterion for function series limits]
    \label{prop:cauchycriterionfunction}
    For $\sum_{k=0}^{\infty}a_k$, the Cauchy criterion is 
    \[
        \forall \eps>0, \exists N\text{ s.t. } n\geq m>N \implies 
        \pa{\sum_{k=m}^{n}a_k}<\eps.
    \]
    For a series of functions,
    \[
        \forall \eps>0, \exists N\text{ s.t. } n\geq m>N, x\in S \implies
        \pa{\sum_{k=m}^{n}g_k(x)}<\eps
    \]
\end{proposition}

\begin{theorem}
[Weierstra\ss $\ M$-test]
    \label{thm:weirstrasmtest}
    Suppose $(M_k)$ is a seq. of non-negative real numbers
    where $\sum_{k=0}^{\infty}M_k$ is finite. If 
    $\pa{g_k(x)}\leq M_k\ \forall x\in S, \forall k$, then 
    $\sum g_k$ converges uniformly on $S$.
\end{theorem}

\begin{proof}
    Since $\sum M_k$ converges, it satisfies \cref{prop:cauchycriterionfunction}:
    \[
        \pa{\sum_{k=m}^{n}M_k}<\eps.
    \]
    Hence 
    \[
        \pa{\sum_{k=m}^{n}g_k(x)}\leq \sum_{k=m}^{n}\pa{g_k(x)}\leq 
        \sum_{k=m}^{n}M_k<\eps.
    \]
    So $g_k$ converges uniformly on $S$.
\end{proof}

\begin{example}
    Use for power series 
    \begin{align*}
        \sum_{n=1}^{\infty}2^{-n}x^n &= x\sum_{n=0}^{\infty}2^{-n-1}x^n \\
        &= \frac{x}{2}\sum_{n=0}^{\infty}\pp{\frac{1}{2}}^nx^n \\
        &= \frac{x}{2}\cdot\frac{1}{1-\frac{x}{2}}=\frac{x}{2-x}.
    \end{align*}
    At $x=\pm2$, the series does not converge. For the interval
    $[-a,a], a<2$, then $\pa{2^{-n}x^n}\leq \pp{\frac{a}{2}}^n$,
    which converges as $n\to \infty$. Now let $M_n=\pp{\frac{a}{2}}^n$. By 
    \cref{thm:weirstrasmtest}, the series converges uniformly on 
    $[-a,a]$. The limit function must be continuous, so it 
    must be continuous on $(-2,2)$.
\end{example}

\clearpage
\section{November 1st, 2022}
\subsection{More power series results}
\begin{theorem}
    Let $\sum_{n=0}^{\infty}a_nx^n$ be a power series with 
    a radius of convergence $R>0$. If $0<R_1<R$, the power series 
    converges uniformly on $[-R_1,R_1]$ to a continuous function.
\end{theorem}

\begin{corollary}
    The power series converges uniformly to a continuous function for the 
    open interval $(-R,R)$.
\end{corollary}
Given a power series, $f(x) = \sum_{n=0}^{\infty}a_nx^n$, then 
$f'(x) = \sum_{n=0}^{\infty}na_nx^{n-1}$ has the same radius 
of convergence, and must converge for the same values of $x$.

\begin{theorem}
[Abel's theorem]
    \label{thm:abelthm}
    Let $f(x)=\sum_{n=0}^{\infty}a_nx^n$ be a power series with a 
    finite positive radius of convergence. If the series 
    converges at $x=R$, then $f$ is continuous at $x=R$,
    and similarly for $x=-R$.
\end{theorem}

\begin{proof}
    Consider $f(x)=\sum_{n=0}^{\infty}a_nx^n$ with radius of convergence 
    $1$ (we will generalize later). Let the series converge at $1$. Let,
    $f_n=\sum_{k=0}^{n}a_kx^k$, and $s_n = f_n(1) = \sum_{k=0}^{n}a_k$.
    We note 
    \[
        \lim s_n=s=\sum_{k=0}^{\infty}a_k=f(1),\quad s_k-s_{k-1}=a_k.
    \]
    For $0<x<1$,
    \begin{align*}
        f_n(x)&=s_0+\sum_{k=1}^{n}(s_k-s_{k-1})x^k \\
        &= s_0+\sum_{k=1}^{n}s_kx^k - x \sum_{k=0}^{n-1}s_kx^k \\
        &= \pp{\sum_{k=0}^{n-1}s_k(1-x)x^k}+s_nx^n.
    \end{align*}
    We note that $f(1)=s$, and $\sum_{n=0}^{\infty}x^n=\frac{1}{1-x}
    \implies \sum_{n=0}^{\infty}(1-x)x^n=1$.
    Taking the limit as $n\to\infty$,
    \begin{align*}
        \lim f_n(x) = f(x) &= \sum_{n=0}^{\infty}s_n(1-x)x^n. \\ 
        f(1) = s &= \sum_{n=0}^{\infty}s(1-x)x^n.\\
        f(1)-f(x) &= \sum_{n=0}^{\infty}(s-s_n)(1-x)x^n.
    \end{align*}
    Choose $\eps>0$. Since $\lim_{n\rightarrow \infty}s_n=s$, 
    $\exists N\in\NN$ s.t. $n>N\implies \pa{s-s_n}<\frac{\eps}{2}$.

    Define $g_N(x)=\sum_{n=0}^{N}\pa{s-s_n}(1-x)x^n$.
    Then 
    \begin{align*}
        \pa{f(1)-f(x)} &\leq g_N(x) + \sum_{n=N+1}^{\infty}\pa{s-s_n}(1-x)x^n\\
        &\leq g_N(x) + \sum_{n=N+1}^{\infty}\frac{\eps}{2}(1-x)x^n \\
        &= g_N(x) + \frac{\eps}{2}\underbrace{\sum_{n=N+1}^{\infty}(1-x)x^n}_{\text{bounded by $1$}}\\
        &< g_N(x)+\frac{\eps}{2}.
    \end{align*}
    $g_N$ is continuous and $g_N(1)=0$. Hence $\exists \delta>0$ s.t. 
    $1-\delta<x<1\implies g_N(x)<\frac{\eps}{2}$.
    Then 
    \[
        \pa{f(1)-f(x)}<\frac{\eps}{2}+\frac{\eps}{2}=\eps.
    \]
    If $f(x)$ has radius of convergence $R$, define $g(x)=f(Rx)$.
    $g$ has radius of convergence $R$, so the proof applies.
    For $-R$, let $h(x)=f(-x)$.
\end{proof}

A consequence is that if a power series and associated function 
agree with the theorem statement, then 
\[
    \lim_{x\rightarrow R^-}f(x) = \sum_{n=0}^{\infty}a_nR^n,
\]
or
\[
    \lim_{x\rightarrow -R^+}f(x) = \sum_{n=0}^{\infty}a_n(-R)^n.
\]

\subsection{Approximating functions}
Any continuous function on $[0,1]$ can be approximated by polynomials, 
they just may not be power series. This can be done in 
terms of \vocab{Bernstein polynomials}. For $f$ continuous on $[0,1]$,
\[
    B_nf(x) = \sum_{k=0}^{\infty}f\pp{\frac{k}{n}}\binom{n}{k}x^k(1-x)^{n-k}.
\]
Although it is not that efficient for some functions, it 
nonetheless can approximate any continuous function.
% \begin{lemma}
%     For $x\in\RR$, $n\geq0$,
%     \[
%         \sum_{k=0}^{n}\binom{n}{k}x^k(1-x)^{n-k}=1.
%     \]
% \end{lemma}

\begin{lemma}
    For $x\in\RR$, $n\geq0$,
    \[
        \sum_{k=0}^{n}(nx-k)^2\binom{n}{k}x^k(1-x)^{n-k}=nx(1-x)\leq \frac{n}{4}.
    \]
\end{lemma}

\begin{proof}
    Note 
    \[
        k\binom{n}{k}=\frac{kn!}{(n-k)!k!} = \frac{n(n-1)!}{(n-k)!(k-1)!}
        =n\binom{n-1}{k-1}.
    \]
    Then
    \begin{align*}
        \sum_{k=0}^{n}k\binom{n}{k}x^k(1-x)^{n-k} &= n \sum_{k=1}^{n}\binom{n-1}{k-1}x^k(1-x)^{n-k} \\
        &= nx \sum_{j=0}^{n-1}\binom{n-1}{j}x^j(1-x)^{n-j-1} \\
        &= nx (x+(1-x))^{n-1} \\
        &= nx.
    \end{align*}
    Similarly, 
    \[
        \sum_{k=0}^{n}k(k-1)\binom{n}{k}x^k(1-x)^{n-k}=n(n-1)x^2.
    \]
    \[
        \sum_{k=0}^{n}k^2\binom{n}{k}x^k(1-x)^{n-k} = n(n-1)x^2+nx = n^2x^2+nx(1-x).
    \]
    Since $(nx-k)^2=n^2x^2-2nxk+k^2$, 
    \begin{align*}
        \sum_{k=0}^{n}(nx-k)^2\binom{n}{k}x^k(1-x)^{n-k} &= n^2x^2-2nxk+k^2 + nx(1-x) \\
        &= nx(1-x)\\
        &\leq \frac{n}{4}.\qedhere
    \end{align*}
\end{proof}

\clearpage
\section{November 3rd, 2022}
\subsection{Finishing Bernstein polynomials}
\begin{theorem}
    \label{thm:bernstein-convergence}
    For every continuous function on $[0,1]$, $B_nf\to f$ uniformly.
    on $[0,1]$.
\end{theorem}
\begin{proof}
    Assume that $f$ is not always $0$. Let 
    $M=\sup\ps{\pa{f(x)}: x\in [0,1]}$.
    Choose $\eps>0$. Then $\exists \delta>0$ s.t. $\pa{x-y}<\delta
    \implies \pa{f(x)-f(y)}<\frac{\eps}{2}$.
    Consider
    $\pa{B_nf(x)-f(x)} = \pa{\sum_{k=0}^{n}f\pp{\frac{k}{n}}\binom{n}{k}x^k(1-x)^{n-k}-f(x)}$. Note that $\sum_{k=0}^{n}\binom{n}{k}x^k(a-x)^{n-k}=1$. We can rewrite the sum as 
    \begin{align*}
        \pa{B_nf(x)-f(x)} &= \pa{\sum_{k=0}^{n}\pp{f\pp{\frac{k}{n}} - f(x)} \binom{n}{k} x^k (1-x)^{n-k}} \\ 
        &\leq \sum_{k=0}^{n}\pa{f\pp{\frac{k}{n}} - f(x)} \binom{n}{k} x^k (1-x)^{n-k}.
    \end{align*}
    We separate the terms in the sum into two cases.
    \begin{itemize}
        \item If $\pa{\frac{k}{n}-x}<\delta$, then 
        $\pa{f\pp{\frac{k}{n}}-f(x)}<\frac{\eps}{2}$.
        \item If $\pa{\frac{k}{n}-x}\geq\delta$, then 
        $\pa{\frac{k-nx}{n}}\geq \delta \implies 
        \frac{(k-nx)^2}{n^2}\geq\delta^2 \implies \pp{k-nx}^2\geq \delta^2n^2$.
    \end{itemize}
    Let $B$ a set of indices where the second case holds.
    \begin{align*}
        \sum_{k\in B}\pa{f\pp{\frac{k}{n}} - f(x)} \binom{n}{k} x^k (1-x)^{n-k}
        &\leq 2M\sum_{k\in B}\binom{n}{k} x^k (1-x)^{n-k} \\
        &\leq \frac{2M}{n^2\delta^2}\sum_{k\in B}(k-xn)^2\binom{n}{k} x^k (1-x)^{n-k} \\
        &\leq \frac{2M}{n^2\delta^2}\cdot\frac{n}{4} \\
        &= \frac{M}{2n\delta^2}.
    \end{align*}
    Consider for $n>N=\frac{M}{\eps\delta^2}$.
    Then 
    \[
        \sum_{k\in B}\pa{f\pp{\frac{k}{n}} - f(x)} \binom{n}{k} x^k (1-x)^{n-k}\leq \frac{M}{2\pp{\frac{M}{\eps\delta^2}}\delta^2} < \frac{\eps}{2}.
    \]
    Let $A$ be the indices where the first case holds.
    Then 
    \begin{align*}
        \sum_{k\in A}
        \pa{f\pp{\frac{k}{n}} - f(x)} \binom{n}{k} x^k (1-x)^{n-k}
        &<
        \sum_{k\in A} \frac{\eps}{2} \binom{n}{k} x^k (1-x)^{n-k}\\
        &\leq 
        \frac{\eps}{2}.
    \end{align*}
    Therefore the entire sum is bounded by 
    $\frac{\eps}{2}+\frac{\eps}{2}=\eps$.
\end{proof}
\begin{theorem}
[Weiersta\ss \ approximation theorem]
    \label{thm:weierstrassapproxthm}
    Every continuous function on a closed interval $[a,b]$  can 
    be uniformly approximated by polynomials on $[a,b]$.
\end{theorem}
For any function $g(x)$ on $[a,b]$, we can create $h(x)$ on $[0,1]$ 
by $h(x)=g(a+(b-a)x)$, and apply \cref{thm:bernstein-convergence} on $h(x)$.

\subsection{Differentiation}

\begin{definition}
    Let $f$ be a real-valued function on $S \subseteq \RR$.
    Define the \vocab{derivative} $f'(a)$ at $a\in S$ as 
    \[
        f'(a) = \lim_{x\rightarrow a}\frac{f(x)-f(a)}{x-a},
    \]
    if it exists and is finite.
\end{definition}
\begin{theorem}
    If $f$ is differentiable at $a$, then $f$ is continuous at $a$.
\end{theorem}
\begin{proof}
    Given 
    \[
        \lim_{x\rightarrow a}\frac{f(x)-f(a)}{x-a}
    \]
    exists, then 
    \[
        f(x) = \underbrace{(x-a)}_{=0}\underbrace{\frac{f(x)-f(a)}{x-a}}_{\text{finite}} + f(a)
    \]
    Therefore 
    \[
        \lim_{x\rightarrow a}f(x) = f(a).\qedhere
    \]
\end{proof}
\begin{proposition}
    Let $f$, $g$ be differentiable at $a$. Let $c\in \RR$.
    Then 
    \begin{enumerate}
        \item $(cf)'(a) = cf'(a)$
        \item $(f+g)'(a) = f'(a) + g'(a)$
        \item $(f\cdot g)'(a) = f(a)g'(a)+f'(a)g(a)$
        \item $(f/g)'(a) = \frac{g(a)f'(a)-f(a)g'(a)}{g(a)^2}$
    \end{enumerate}
    except when $g(a)=0$ for $f/g$.
\end{proposition}
\begin{proof}
    [Proof for (3)]
    \begin{align*}
        \frac{(fg)(x)-(fg)(a)}{x-a} &= \frac{f(x)g(x)-f(x)g(a)+f(x)g(a)-f(a)g(a)}{x-a} \\
        &= f(x)\frac{g(x)-g(a)}{x-a} + g(a)\frac{f(x)-f(a)}{x-a}.
    \end{align*}
    Therefore 
    \[
        \lim_{x\rightarrow a}\frac{(fg)(x)-(fg)(a)}{x-a} = f(a)g'(a) + g(a)f'(a). \qedhere
    \]
\end{proof}
\begin{theorem}
[Chain rule]
    \label{thm:chain-rule}
    Suppose that $f$ differentiable at $a$ and $g$ differentiable at $f(a)$.
    Then 
    \[
        (g\circ f)'(a) = g'(f(a))f'(a).
    \]
\end{theorem}
\begin{proof}
    Define
    \[
        h(y) = \frac{g(y)-g(f(a))}{y-f(a)}
    \]
    for $y\in \dom(g), y\neq f(a)$.
    Also define $h(f(a)) = g'(f(a))$. Consider 
    \[
        \lim_{y\rightarrow f(a)}h(y) = h(f(a)) = g'(f(a)).
    \]
    Hence $h$ is continuous at $f(a)$. Now 
    \[
        g(y) - g(f(a)) = h(y)(y-f(a)) \quad \forall y\in\dom(g).
    \]
    Let $y=f(x), x\in\dom(g\circ f)$.
    \[
        (g\circ f)(x) - (g\circ f)(a) = h(f(x))(f(x)-f(a))
    \] 
    \[
        \frac{(g\circ f)(x) - (g\circ f)(a)}{x-a} = \frac{h(f(x))(f(x)-f(a))}{x-a}.
    \]
    Taking the limit,
    \[
        (g\circ f)'(a) = g'(f(a))f'(a). \qedhere
    \]
\end{proof}
\clearpage
\section{November 15th, 2022}
\subsection{Derivative properties}
\begin{theorem}
    If $f$ is defined on an open interval $x_0$, and $f$ assumes 
    a min/max at $x_0$, and $f$ is differentiable at $x_0$,
    then $f'(x_0) = 0$.
\end{theorem}
\begin{proof}
    Suppose $f$ is defined on $(a,b)$ with $a<x_0<b$. Suppose 
    the max is at $x_0$.
    If $f'(x_0)>0$, then 
    \[
        f'(x_0) = \lim_{x\rightarrow x_0}\frac{f(x)-f(x_0)}{x-x_0}>0.
    \]
    Pick $\eps = f'(x_0)$. $\exists \delta$ s.t. $0<\pa{x-x_0}<\delta$,
    then 
    \[
        \pa{\frac{f(x)-f(x_0)}{x-x_0}-f'(x_0)} < f'(x_0).
    \]
    \[
        -f'(x_0)< \frac{f(x)-f(x_0)}{x-x_0}-f'(x_0)< f'(x_0).
    \]
    \[
        0< \frac{f(x)-f(x_0)}{x-x_0}.
    \]
    Thus there is a point $x \in (x_0,x_0+\delta)$
    such that $f(x)>f(x_0)$.
\end{proof}
\begin{theorem}
    [Rolle's theorem]
    \label{thm:rollesthm}
    Let $f$ be a continuous function on $[a,b]$ that is differentiable on 
    $(a,b)$ and satisfies $f(a)=f(b)$.  Then $\exists x\in(a,b)$ 
    s.t. $f'(x) = 0$.
\end{theorem}
\begin{proof}
    By a previous theorem, $f$ is bounded and achieves its bounds.
    Then $\exists x_0,y_0\in[a,b]$ s.t. $f(x)\in [f(x_0),f(y_0)]
    \forall x\in[a,b]$.

    Clearly if $x_0$ and $y_0$ are both endpoints, then $x_0=y_0$,
    and $f(x) = x_0$ is constant.

    Otherwise, $f$ assumes a maximum/minimum at $x\in(a,b)$,
    in which case $f'(x)=0$.
\end{proof}
This can be generalized to,
\begin{theorem}
    [Mean value thoerem]
    \label{thm:mvt}
    Let $f$ be continuous on $[a,b]$ and differentiable on $(a,b)$.
    Then $\exists x\in(a,b)$ s.t. 
    \[
        f'(x) = \frac{f(b)-f(a)}{b-a}.
    \]
\end{theorem}
\begin{proof}
    Let $L(x) = f(a) + \frac{(x-a)}{b-a}(f(b)-f(a))$.
    Then $L'(x) = \frac{f(b)-f(a)}{b-a}$. Let $g(x) = f(x) - L(x)$ and 
    $g(a)=0=g(b)$. Rolle's theorem tells us that $g'(x)=0$, so 
    $f'(x) = L'(x) = \frac{f(b)-f(a)}{b-a}$.
\end{proof}
\begin{corollary}
    Let $f$ be differentiable on $(a,b)$ s.t. $f'(x) = 0, \forall x \in(a,b)$. Then $f$ is a constant function on $(a,b)$. 
\end{corollary}
\begin{proof}
    If non-constant, $\exists x_1,x_2$ s.t. $a<x_1<x_2<b$ and 
    $f(x_1)\neq f(x_2)$. Applying \cref{thm:mvt} to $x_1,x_2$,
    $\exists x\in (x_1,x_2)$ s.t. $f'(x) = \frac{f(x_1)-f(x_2)}{x_1-x_2}$ 
    is non-zero, contradicting our assumption.
\end{proof}
\begin{corollary}
    If $f$, $g$ are differentiable on $(a,b)$ and $f'=g'$ on 
    $(a,b)$, then $\exists c\in\RR$ s.t. $f(x)=g(x) + c, \forall x\in(a,b)$.
\end{corollary}
\begin{proof}
    Apply the previous corollary to $f-g$. Then $f(x)-g(x) = c\in\RR$.
    Then $f(x) = g(x)+C$.
\end{proof}

\begin{corollary}
    Let $f$ be differentiable on $(a,b)$. Then 
    \begin{itemize}
        \item $f$ is strictly increasing if $f'(x)>0 \forall x \in (a,b)$
        \item $f$ is increasing if $f'(x)\geq 0 \forall x \in (a,b)$
    \end{itemize}    
\end{corollary}

\begin{proof}
    Consider $x_1,x_2$ s.t. $a<x_1<x_2<b$. \cref{thm:mvt} tells us 
    that $\exists x$ s.t. $\frac{f(x_2)-f(x_1)}{x_2-x_1} = f'(x) > 0$.
\end{proof}

\begin{theorem}
[IVT for derivatives]
    \label{thm:ivtderivatives}
    Let $f$ be differentiable on $(a,b)$. Whenever $a<x_1<x_2<b$ and 
    $c$ is between $f'(x_1), f'(x_2)$, $\exists x\in(x_1,x_2)$ s.t.
    $f'(x)=c$.
\end{theorem}

\begin{proof}
    WLOG $f'(x_1)<c<f'(x_2)$. Let $g(x) = f(x)-cx$ for $x\in (a,b)$.
    Then $g'(x_1)<0<g'(x_2)$. A previous theorem tells us that $g$ 
    assumes its minimum on $x_0\in[x_1,x_2]$.
    \[
        g'(x_1) = \lim_{y\rightarrow x_1}\frac{g(y)-g(x_1)}{y-x_1}<0.
    \]
    Hence $g(y)-g(x_1)<0$ close to $x_1$ or $g(y)<g(x_1)$.
    Therefore $x_1$ is not a minimum. Similarly, we can show $x_2$ is 
    not a minimum. $g'(x_0)=0$ by a previous theorem, 
    so $f'(x_0) = g'(x_0) + c = c$.
\end{proof}
\clearpage
\section{November 17th, 2022}
\subsection{Exam review}
\begin{ques}
    [Exam Question 5]
    Consider the function defined on $[0,\infty)$ as 
    \[
        g(x) = \begin{cases}
            x(1-x) &\text{if }0\leq x\leq 1 \\
            0 &\text{if }x>1
        \end{cases}
    \]
    Let $f_n(x)=g(nx)$.
    \begin{enumerate}[a)]
        \item What is the maximum value of $g$ and where is it attained?
        \item Sketch the functions $f_1(x), f_2(x), f_3(x)$ on $[0,1]$.
        \item Does $f_n$ pointwise converge to some $f$.
        \item Does $f_n \to f$ uniformly?
    \end{enumerate}
\end{ques}
Only 50\% of the class got (b)'s sketch correctly.
No one got (c), (d) w/o the correct sketch for (b).
\begin{soln}
    Taking $g(x) \mapsto g(2x)$ is the same as geometrically scaling it by 
    $\frac{1}{2}$ along the $x$-axis. 
    These sketches allow the rest of the problem 
    to be finished.
\end{soln}
\subsection{More differentiation}
Let $f$ be an injective differentiable on an interval $I$.
We try to conclude that $(f\inv)'(y_0) = \frac{1}{f'(x_0)}$.
However, this requires assuming $(f\inv)'$ exists.

\begin{theorem}
    \label{thm:funcinvderivative}
    Let $f$ be an injective continuous function on an open interval $I$, 
    and let $J=f(I)$. If $f$ is differentiable at $x_0\in I$ and 
    $f'(x_0) \neq 0$, then $f\inv$ is differentiable at $y_0=f(x_0)$, and 
    \[
        (f\inv)'(y_0) = \frac{1}{f'(x_0)}.
    \]
\end{theorem}

\begin{proof}
    $J$ is an open interval as well, so 
    \[
        \lim_{x\rightarrow x_0}\frac{f(x)-f(x_0)}{x-x_0}=f'(x_0).
    \]
    Choose $\eps>0$. $\exists \delta >0$ s.t. 
    \[
        0 < \pa{x-x_0} < \delta \implies \pa{\frac{x-x_0}{f(x)-f(x_0)}-\frac{1}{f'(x_0)}} < \eps.
    \]
    Let $g=f\inv$. $g$ is continuous at $y_0$. $\exists \eta > 0$ s.t.
    \[
        0 < \pa{y-y_0} < \eta \implies \pa{g(y)-g(y_0)} < \delta
        \implies \pa{g(y) - x_0} < \delta.
    \]
    Hence 
    \[
        \pa{\frac{g(y)-x_0}{f(g(y)) - f(x_0)} - \frac{1}{f'(x_0)}} = \pa{\frac{g(y)-g(y_0)}{f(g(y)) - f(x_0)} - \frac{1}{f'(x_0)}} < \eps.
    \]
    for arbitrarily small $\eps$. Therefore 
    \[
        \lim_{y\rightarrow y_0}\frac{g(y)-g(y_0)}{y-y_0} = \frac{1}{f'(x_0)}. \qedhere
    \]
\end{proof}
\subsubsection{L'H\^opital's rule}
Consider a generic limit 
\[
        \lim_{x\rightarrow s}\frac{f(x)}{g(x)},
    \]
where the limit of $f$ and $g$ are both $0$.
\begin{example}
    [$\mathrm{sinc}$ function]
    $\mathrm{sinc}\ x = \frac{\sin x}{x}$ satisfies these 
    properties. We can find $\lim_{x\to 0}\mathrm{sinc}\ x=1$.
\end{example}
\begin{theorem}
    [Generalized MVT]
    \label{thm:generalizedmvt}
    Let $f$, $g$ be continuous on $[a,b]$ and differentiable on $(a,b)$.
    Then $\exists x \in (a,b)$ s.t. $f'(x)(g(b)-g(a)) = g'(x)(f(b)-f(a))$.
\end{theorem}
Note that if we let $g(x)=x$, then get the original MVT.
\begin{proof}
    Let 
    \[
        h(x) = f(x)(g(b)-g(a)) - g(x)(f(b)-f(a)).
    \] 
    \[
        h(a) = f(a)g(b) - f(b)g(a) = h(b).
    \]
    $h$ satisfies \cref{thm:rollesthm}, so $\exists x\in (a,b)$ 
    s.t. $h'(x) = 0$. The result follows from considering $h'(x)$.
\end{proof}
\begin{theorem}
    [L'H\^opital's rule]
    \label{thm:lopital}
    Suppose $f,g$ are differentiable, let $s$ be any limit. Suppose
    \[
        \lim_{x\rightarrow s}\frac{f'(x)}{g'(x)} = L
    \]
    exists. If $\lim_{x\rightarrow s}f(x) = \lim_{x\rightarrow s}g(x) = 0$,
    or $\lim_{x\rightarrow s}\pa{g(x)} = \infty$, then 
    \[
        \lim_{x\rightarrow s}\frac{f(x)}{g(x)} = L.
    \]
\end{theorem}
\begin{proof}
    Consider $\lim_{n\rightarrow a^+}$ or $\lim_{x\rightarrow -\infty}$.
    We will show 
    \begin{claim}
        If $-\infty\leq L< \infty$ and $L_1> L$
        $\exists \alpha_1>a$ s.t. $a<x<\alpha_1 \implies \frac{f(x)}{g(x)}<L_1$.
    \end{claim}
    \begin{subproof}
        Let $(a,b)$ be an interval on which $f$ and $g$ are differentiable
        and on which $g'$ never vanishes.

        Either $g'(x) > 0 \ \forall x\in(a,b)$ or $g'(x) < 0 \ \forall x\in[a,b]$, which follows from IVT for derivatives.

        Assume $g'(x)<0$. Then $g$ is strictly decreasing and injective.
        $g(x) = 0$ for at most one $x\in (a,b)$.
        We choose $b$ smaller than this value to ensure $g$ does not 
        vanish.

        Choose $L<K<L_1$. $\exists \alpha$ s.t. $a<x<\alpha\implies \frac{f'(x)}{g'(x)}<K$.

        If $a<x<y<\alpha$, $\exists z\in(x,y)$ s.t. 
        \[
            \frac{f(x)-f(y)}{g(x)-g(y)} = \frac{f'(z)}{g'(z)} < K
        \]
        \begin{itemize}
            \item \textbf{Case 1:} $\lim_{x\rightarrow a^+}f(x)=\lim_{x\rightarrow a^+}g(x) = 0$.
            \[
                \lim_{x\rightarrow a^+}\frac{f(x)-f(y)}{g(x)-g(y)} = \frac{f(y)}{g(y)} \leq K < L_1.
            \]
            \item \textbf{Case 2:} $\lim_{x\rightarrow a^+}g(x) = \infty$.
            We multiply an above expression by $\frac{g(x)-g(y)}{g(x)}$ to 
            get 
            \[
                \frac{f(x)}{g(x)} < K + \frac{f(y)-Kg(y)}{g(x)}.
            \]
            Note 
            \[
                \lim_{x\rightarrow a^+}\frac{f(y)-Kg(y)}{g(x)} = 0.
            \]
            So $\exists \alpha_2>a$ s.t. $a<x<\alpha_2$ and $\frac{f(x)}{g(x)}<L_1$. \qedhere
        \end{itemize}
    \end{subproof}
    \begin{claim}
        If $-\infty<L\leq\infty$ and $L_2<L$, $\exists \alpha_2>a$ s.t.
        $a < x < \alpha_2 \implies \frac{f(x)}{g(x)}>L_2$.
    \end{claim}
    \begin{subproof}
        Similar to the last proof.
    \end{subproof}
    Suppose $L$ is finite. Then 
    \[
        \exists a< x< \alpha_1 \implies \frac{f(x)}{g(x)}<L+\eps,
    \]
    \[
        \exists a< x< \alpha_2 \implies \frac{f(x)}{g(x)}<L-\eps.
    \]
    Define $\alpha=\min\ps{\alpha_1,\alpha_2}$.
    \[
        a < x < \alpha \implies \pa{\frac{f(x)}{g(x)}-L} < \eps,
    \]
    finishing our proof.
\end{proof}
\clearpage
\section{November 22nd, 2022}
\subsection{Prerequisites for Taylor series}
\begin{theorem}
    Suppose $\sum_{n=0}^{\infty}c_nx^n$ converges for $\pa{x}<R$ 
    and define $f(x)=\sum_{n=0}^{\infty}c_nx^n$ for $\pa{x}<R$.
    Then $f'(x) = \sum_{n=1}^{\infty}nc_nx^{n-1}$.
\end{theorem}
\begin{proof}
    Use $\limsup \pa{c_nn}^{1/n} = \limsup \pa{c_n}^{1/n}$.
\end{proof}
\begin{theorem}
    Suppose $\ps{f_n}$ is a seq. of differentiable functions on $[a,b]$ s.t.
    $\ps{f_n(x_0)}$ converges for $x_0\in [a,b]$. If $\ps{f_n'}$ converges 
    uniformly on $[a,b]$, then $f_n$ converges uniformly on $[a,b]$ 
    to $f$ and $f'(x) = \lim_{n\rightarrow \infty}f_n'(x)$.
\end{theorem}
\begin{proof}
    Choose $\eps>0$ and $N$ s.t. $m,n>N$ implies 
    \[
        \pa{f_n(x_0)-f_m(x_0)} < \frac{\eps}{2},
    \]
    and 
    \[
        \pa{f_n'(t)-f_m'(t)} < \frac{\eps}{2(b-a)}, \text{ such that } t\in [a,b].
    \]
    Apply \cref{thm:mvt} to $f_n-f_m$. Thus $\exists y_0 \in (x,t)$
    s.t. 
    \[
        \frac{\pa{f_n(x)-f_m(x) - f_n(t)+f_m(t)}}{\pa{x-t}} = \pa{f_n'(y_0)-f_m'(y_0)}.
    \]
    Thus 
    \[
        \pa{f_n(x)-f_m(x) - f_n(t)+f_m(t)} \leq \frac{\pa{x-t}\eps}{2(b-a)} < \frac{(b-a)\eps}{2(b-a)} = \frac{\eps}{2}.
    \]
    \[
        \pa{f_n(x)-f_m(x)} \leq \pa{f_n(x)-f_m(x) - f_n(x_0)+f_m(x)} + \pa{f_n(x)-f_m(x_0)} < \frac{\eps}{2}+\frac{\eps}{2}.
    \]
    $f_n$ converges uniformly. Let $f(x)=\lim_{n\rightarrow \infty}f_n(x)$.

    For the next part, consider $x,t \in [a,b]$ $x\neq t$.
    \[
        \phi_n(t) = \frac{f_n(t)-f_n(x)}{t-x} \qquad \phi(t) = \frac{f(t)-f(x)}{t-x}.
    \]
    $\lim_{t\rightarrow x}\phi_n(t) = f_n'(x)$. Since $f_n\to f$ uniformly,
     $\phi_n$ converges uniformly. $\exists N$ s.t. $n,m>N\implies $
    \[
        \pa{\phi_n(t)-\phi_m(t)} < \frac{\eps}{2(b-a)},
    \]
    and $\lim_{t\rightarrow x}\phi_n(t) = \phi(t)$.
    This is the same setup as the last part, so we can conclude that 
    \[
        \lim_{t\rightarrow x}\phi(t) = \lim_{n\rightarrow \infty}f_n'(x)=f'(x).
        \qedhere
    \]
\end{proof}
\subsection{Taylor series}
Consider the power series $f(x)=\sum_{k=0}^{\infty}a_kx^k$.
Then $f'(x) = \sum_{k=1}^{\infty}ka_kx^{k-1}$, and generally,
$f^{(n)}(x) = \sum_{k=n}^{\infty}k(k-1)\cdots(k-n+1)a_kx^{k-n}$.
Then $f^{(n)}(0) = n!a_n$. This motivates,
\begin{definition}
    The \vocab{Taylor series of $f$ about $0$} is the power series 
    \[
        f(x) = \sum_{k=0}^{\infty}\frac{f^{(k)}(0)}{k!}x^k.
    \]
    The \vocab{remainder} is defined as 
    \[
        R_n(x) := f(x) - \sum_{k=0}^{n-1}\frac{f^{(k)}(0)}{k!}x^k
    \]
\end{definition}
$f$ equal its Taylor series if and only if 
$\lim_{n\rightarrow \infty}R_n(x) = 0$.
\begin{theorem}
[Taylor's theorem]
    \label{thm:taylorthm}
    Let $f$ be defined on $(a,b)$ where $a<0<b$ and suppose 
    the $n$th derivative $f^{(n)}$ exists on $(a,b)$.
    Then for each non-zero $x\in (a,b)$, $\exists y$ between $0$ and $x$ 
    s.t. $R_n(x)=\frac{f^{(n)}(y)x^n}{n!}$.
\end{theorem}
\begin{proof}
    Fix $x\neq 0$. Assume that $x>0$. Let $M$ be the unique sol'n of 
    \[
        f(x) = \sum_{k=0}^{n-1}\frac{f^{(k)}(0)}{k!}x^k + \frac{Mx^n}{n!}.
    \]
    Let 
    \[
        g(t) = \sum_{k=0}^{n-1}\frac{f^{(k)}(0)}{k!}t^k + \frac{Mt^n}{n!}-f(t).
    \]
    $g(0)=f(0)-f(0)=0$, and $g^{(k)}(0)=0$ for $k<n$. In addition, $g(x)=0$.
    By \cref{thm:rollesthm}, $\exists x_1\in(0,x)$ s.t. $g'(x_1)=0$.
    We can apply Rolle's theorem again, so $\exists x_2\in(0,x_1)$ s.t.
    $g''(x_2)=0$. We can recursively apply this to find 
    $x_n\in (0,x_{n-1})$ s.t. $g^{(n)}(x_n)=0$.

    Then $g^{(n)}(t) = \frac{n!M}{n!}-f^{(n)}(t)$.
    $f^{(n)}(x) = M$, proving the result.
\end{proof}
\begin{corollary}
    Let $f$ be defined on $(a,b)$ where $a<0<b$. If all derivatives 
    $f^{(n)}$ exist on $(a,b)$ and are bounded by a single $C$,
    then $\lim_{n\rightarrow \infty}R_n(x) = 0$ for all $x\in(a,b)$.
\end{corollary}
\begin{proof}
    $\pa{R_n(x)}\leq \frac{C}{n!}\pa{x}^n$. $\lim_{n\rightarrow \infty}R_n(x)=0$.
\end{proof}
\begin{definition}
    Let $f$ be a function on an interval containing $x_0\in\RR$. If $f$ 
    has derivatives of all order at $x_0$, then 
    \[
        \sum_{k=0}^{\infty}\frac{f^{(k)}(x_0)}{k!}(x-x_0)^k,
    \]
    is the \vocab{Taylor series of $f$ about $x_0$}.
\end{definition}
\subsection{Riemann integration}
Consider a bounded function on a closed interval $[a,b]$. For 
$S \subseteq [a,b]$, define 
\[
    M(f,S) = \sup\ps{f(x)\mid x\in S}, \qquad m(f,S) = 
    \inf\ps{f(x)\mid x\in S}.
\]
Define a partition of $[a,b]$ as any finite ordered subset 
\[
    P = \ps{a=t_0<t_1<\cdots<t_n=b}.
\]
The \vocab{upper Darboux sum} $U(f,P)$ of $f$ w.r.t. $P$ is 
\[
    U(f,P) = \sum_{k=1}^{n}M(f,[t_{k-1},t_k])(t_k-t_{k-1}),
\]
and the \vocab{lower Darkboux sum} is 
\[
    L(f,P) = \sum_{k=1}^{n}m(f,[t_{k-1},t_k])(t_k-t_{k-1}).
\]
We see that 
\[
    m(f,[a,b])(b-a) \leq L(f,P) \leq U(f,P) \leq M(f,[a,b])(b-a).
\]
Define integrals as 
\[
    U(f) = \inf\ps{U(f,P)\mid P \text{ is a partition of }[a,b]},
\]
\[
    L(f) = \sup\ps{L(f,P)\mid P \text{ is a partition of }[a,b]}.
\]
\clearpage
\section{November 29th, 2022}

\subsection{Darboux integrals}

The bounds imply that $U(f)$ and $L(f)$ are real numbers, and 
$L(f)\leq U(f)$. We say $f$ is \vocab{integrable} on $[a,b]$ if 
$L(f)=U(f)$.
Then 
\[
    \int_a^b f = \int_a^b f(x)dx = L(f) = U(f).
\]
We call this the \vocab{Darboux integral}.

\begin{example}
    [Darboux integration proof]
    Consider $f(x)=x^3$ and $\int_0^bf$. For the partition 
    $P=\ps{0=t_0<t_1<\cdots<t_n=b}$. Define $t_n = \frac{kb}{n}$.

    Then 
    \begin{align*}
        U(f,P) &= \sum_{k=1}^{n}t_k^3(t_k-t_{k-1})\\ 
        &= \frac{b^4}{n^4}\sum_{k=1}^{n}k^3\\ 
        &= \frac{b^4}{n^4}\pp{\frac{n(n+1)}{2}}^2 \\
        &= \frac{b^4}{4}\pp{1+\frac{2}{n}+\frac{1}{n^2}}.
    \end{align*}
    And 
    \begin{align*}
        L(f,p) &= \sum_{k=1}^{n}t_{k-1}^3(t_k-t_{k-1}) \\
        &= \frac{b^4}{n^4}\sum_{k=1}^{n}(k-1)^3 \\
        &= \frac{b^4}{n^4}\sum_{\ell=1}^{n-1}\ell^3 \\
        &= \frac{b^4}{n^4}\pp{\frac{(n-1)n}{2}}^2 \\
        &= \frac{b^4}{4}\pp{1-\frac{2}{n}+\frac{1}{n^2}}.
    \end{align*}
    As $n\to\infty$, $U(f,P)\to \frac{b^4}{4}$, so $U(f)\leq\frac{b^4}{4}$. 
    Similarly, $L(f)\geq \frac{b^4}{4}$. Therefore $L(f)=U(f) = \frac{b^4}{4}$, and $f$ integrable on $[0,b]$. 
\end{example}

\begin{lemma}
    For partitions $P,Q$ of $[a,b]$ s.t. $P \subseteq Q$,
    \[
        L(f,P) \leq L(f,Q) \leq U(f,Q) \leq U(f,P).
    \]
\end{lemma}

\begin{proof}
    WLOG suppose $P = \ps{a=t_0<t_1<\cdots<t_n = b}$, and \\
    $Q = \ps{a=t_0<t_1<\cdots<t_{k-1}<u<t_k<\cdots<t_n = b}$.
    Then 
    \begin{align*}
        L(f,Q) - L(f,P) &= m(f, [t_{k-1},u])(u-t_{k-1}) + m(f, [u,t_{k}])(t_{k}-u)\\ &\quad- m(f,[t_{k-1},t_k])(t_k-t_{k-1}).
    \end{align*}
    Note 
    \begin{align*}
        m(f,[t_{k-1},t_k])(t_k-t_{k-1}) &= m(f,[t_{k-1},t_k])(t_k-u) + m(f,[t_{k-1},t_k])(u-t_{k-1}) \\
        &\leq m(f, [t_{k-1},u])(u-t_{k-1}) + m(f, [u,t_{k}])(t_{k}-u).
    \end{align*}
    Hence $L(f,Q)-L(f,P)\geq 0$.
\end{proof}

\begin{lemma}
    For partitions $P,Q$ on $[a,b]$, $L(f,P)\leq U(f,Q)$.
\end{lemma}
\begin{proof}
    $P\cup Q$ is also a partition of $[a,b]$ yielding 
    \[
        L(f,P) \leq L(f,P\cup Q) \leq  U(f,P\cup Q) \leq U(f,Q).\qedhere
    \]
\end{proof}
\begin{proposition}
    $L(f)\leq U(f)$.
\end{proposition}

\begin{proof}
    This follows from the properties of limits.
\end{proof}

\begin{theorem}
    [Integrable $\eps$ proof]
    A bounded function $f$ on $[a,b]$ is integrable iff $\forall \eps>0, \exists$ a partition $P$ s.t. $U(f,P)-L(f,P)<\eps$. 
\end{theorem}
\begin{proof}
    Suppose $f$ integrable. $\exists$ $P_1,P_2$ s.t. 
    \[
        L(f,P_1) > L(f)-\frac{\eps}{2},
    \]
    \[
        L(f,P_2) < U(f)+\frac{\eps}{2}.
    \]
    For $P = P_1\cup P_2$, 
    \[
        U(f,P)-L(f,P) \leq U(f,P_2)-L(f,P_1) < U(f) - L(f) + \eps.
    \]
    Since $f$ is integrable, $U(f) = L(f)$, leading to the conclusion.

    Consversely, suppose that $\exists \eps>0$ s.t. the statment holds.
    Then 
    \[
        U(f) \leq U(f,P) = U(f,P) - L(f,P) + L(f,P) < \eps + L(f).
    \]
    Since $\eps$ is arbitrary, $U(f)\leq L(f)\implies U(f)=L(f)$.
\end{proof}
\begin{example}
    [Non-Riemann integrable function]
    $1_\QQ$ on $[0,1]$ has all $U(f) = 1, L(f) = 0$, since rationals and irrationals are dense in the reals.
\end{example}
\begin{definition}
    The \vocab{mesh} of a partition $P$ is the maximum length 
    of subintervals comprising $P$.
    If $P = \ps{a=t_0<t_1<\cdots< t_n =b}$, then 
    \[ \mathrm{mesh}(P) = \max_{1\leq k\leq n}\ps{t_k-t_{k-1}}. \]
\end{definition}
\begin{theorem}
    [Integrable $\delta$-$\eps$ proof]
    \label{thm:darbouxdeltaeps}
    A bounded function $f$ on $[a,b]$ is integrable iff 
    for each $\eps>0$, $\exists \delta>0$ s.t. 
    $\mathrm{mesh}(P)<\delta\implies U(f,P)-L(f,P)<\eps$ for all 
    partitions $P$.
\end{theorem}
\begin{proof}
    Suppose $f$ is integrable. Let $\eps>0$ and $P_0$ is a partition of $[a,b]$ s.t. 
    \[
        U(f,P_0) - L(f,P_0) < \frac{\eps}{2}.
    \]
    Since $f$ is bounded, $\exists B>0$ s.t. $\pa{f(x)} \leq B \forall x\in[a,b]$. Let $\delta = \frac{\eps}{8mB}$ where $m$ is the number 
    of intervals on $P_0$.

    Let $P$ be a partiton of $[a,b]$ with $\mathrm{mesh}(P)<\delta$.
    Let $Q = P\cup P_0$. If $Q$ has one more element than $P$,
    \[
        L(f,Q) - L(f,P) \leq B \mathrm{mesh}(P) - (-B)\mathrm{mesh}(P) = 2B \mathrm{mesh}(P).
    \]
    $Q$ has at most $m$ elements not in $P$, hence 
    \[
        L(f,Q) - L(f,P) \leq2mB\cdot\mathrm{mesh}(P) = \frac{\eps}{4}.
    \]
    Then $L(f,P_0)-L(f,P)<\frac{\eps}{2}$. Similarly $U(f,P_0)-U(f,P)<\frac{\eps}{4}$. 
    \[
        U(f,P)-L(f,P) < U(f,P_0)-L(f,P_0) + \frac{\eps}{2} < \eps.
    \]
    Converse follows easily from definitions.
\end{proof}
These two theorems give ways to show specific properties of integrable 
(bounded) functions.
\begin{theorem}
    Every continuous function on $[a,b]$ is integrable.
\end{theorem}
\begin{proof}
    Consider $\eps>0$. Since $f$ in \textit{uniformly} continuous on $[a,b]$, $\exists \delta>0$ s.t. $\pa{x-y}<\delta \implies \pa{f(x)-f(y)}<\frac{\eps}{b-a}$. 
    
    Consider a partition $P = \ps{a=t_0<\cdots<t_n=b}$
    s.t. $\mathrm{mesh}(P) < \delta$.
    Within any interval $[t_{k-1},t_k]$, $\pa{f(x)-f(y)}<\frac{\eps}{b-a}$.
    Hence 
    \[
        M(f,[t_{k-1},t_k]) - m(f,[t_{k-1},t_k]) < \frac{\eps}{b-a}.
    \]
    So $U(f,P)-L(f,P)<\sum_{k=1}^{n}\frac{\eps}{b-a}(t_k-t_{k-1}) = \eps$.
\end{proof}
\clearpage
\section{December 1st, 2022}
\subsection{Darboux integration results}
\begin{theorem}
    Every monotonic function $f$ on $[a,b]$ is integrable.
\end{theorem}
\begin{proof}
    WLOG assume $f$ increasing and $f(a)<f(b)$.
    $f$ is bounded on $[a,b]$. Choose $\eps>0$, and a partition 
    $P = \ps{a=t_0<\cdots<t_n=b}$ with $\mathrm{mesh}(P) < \frac{\eps}{f(b)-f(a)}$. Then 
    \[
        U(f,P) - L(f,P) = \sum_{k=1}^{n}(M(f, [t_{k-1},t_k])-m(f, [t_{k-1},t_k]))(t_k - t_{k-1})
    \]
    $M(f, [t_{k-1},t_k])-m(f, [t_{k-1},t_k])$ gets simplified to $f(t_k)-f(t_{k-1})$ since $f$ is increasing.
    \begin{align*}
        U(f,P) - L(f,P) &= \sum_{k=1}^{n}(f(t_k)-f(t_{k-1}))(t_k - t_{k-1}) \\
        &< \frac{\eps}{f(b)-f(a)} \sum_{k=1}^{n}(f(t_k)-f(t_{k-1}))\\
        &= \eps. \qedhere
    \end{align*}
\end{proof}
\subsection{Riemann sums and integrals}

\begin{definition}
    Let $f$ be bounded function on $[a,b]$, and $P=\ps{a=t_0<\cdots<t_n=b}$
    be a partition. A \vocab{Riemann sum} of $f$ associated with $P$
    is a sum of the form 
    \[
        \sum_{k=1}^{n}f(x_k)(t_k-t_{k-1}),
    \]
    where $x_k\in[t_{k-1},t_k]$.

    A function is \vocab{Riemann integrable} on $[a,b]$ if $\exists r$
    s.t. $\forall \eps>0$, $\exists \delta>0$ s.t.
    \[
        \pa{S-r}<\eps \text{ for every Riemann sum $S$ of $f$ s.t. $\mathrm{mesh}(P)<\delta$}.
    \]
    If so, $r$ is the Riemann integral of $f$ on $[a,b]$, denoted 
    $\mathscr{R}\int_a^b$.
\end{definition}
The next result shows that Riemann integration is just 
Darboux integration.

\begin{theorem}
    A bounded function $f$ on $[a,b]$ is Riemann integrable iff it is Darboux integrable.
\end{theorem}

\begin{proof}
    ($\implies$) 
    Suppose $f$ Darboux integrable. Let $\eps>0$, and choose $\delta$ s.t.
    \cref{thm:darbouxdeltaeps} is true.
    We have 
    \[
        L(f,P) \leq S \leq U(f,P).
    \]
    \[
        U(f,P) < L(f,P)+\eps \leq L(f)+\eps = \int_a^b f + \eps,
    \]
    and 
    \[
        L(f,P) > U(f)-\eps = \int_a^b f - \eps.
    \]
    Therefore 
    \[
        \int_a^b f -\eps < S < \int_a^b f+ \eps \implies \pa{S - \int_a^bf}< \eps.
    \]
    ($\ \Longleftarrow\ $) Suppose $f$ is Riemann integrable. Consider $\eps>0$. $\exists \delta$ and $r$ s.t. $\mathrm{mesh}(P)<\delta$
    and Riemann sum $S$, 
    \[
        \pa{S-r}<\eps.
    \]
    Choose $x_k \in [t_{k-1},t_k]$ s.t. $f(x_k) < m(f, [t_{k-1}, t_k]) + \eps$. Do this for all intervals.
    \[
        S < L(f,P) + \eps(b-a).
    \]
    \[
        L(f) \geq L(f,P) \geq S - \eps(b-a) > r-\eps-\eps(b-a).
    \]
    Since $\eps$ is arbitrary, $L(f)\geq r$. Similarly, $U(f)\leq r$.
    Then $L(f)=U(f)$, so $f$ is Darboux integrable.
\end{proof}

\begin{proposition}
    [Riemann/Darboux integration properties]
    \label{prop:riemannintegrationprop}
    If $f$ and $g$ are integrable functions on $[a,b]$,
    \begin{enumerate}
        \item $cf$ is integrable, and $\int_a^b cf = c\int_a^b f$.
        \item $f+g$ is integrable, and $\int_a^b f+g = \int_a^b f+\int_a^b g$.
    \end{enumerate}
\end{proposition}

\begin{proof}
    (1) Suppose $c>0$. For a given partition, 
    \[
        M(cf, [t_{k-1},t_k]) = cM(f,[t_{k-1},t_k]).
    \] 
    Hence $U(cf,P) = cU(f,P)$. Therefore $U(cf) = cU(f)$.
    Similarly, $L(cf)=cL(f)$. This shows $\int_a^b cf = c\int_a^b f$.
    For negative $c$, you would have to flip $U$ and $L$.

    (2) Choose $\eps>0$. $\exists$ partitions $P_1,P_2$ s..t 
    \[
        U(f,P_1) - L(f,P_1) < \frac{\eps}{2}, \qquad U(g,P_2) - L(g,P_2) < \frac{\eps}{2}.
    \]
    Let $P = P_1\cup P_2$,
    \[
        U(f,P)-L(f,P) < \frac{\eps}{2}, \qquad U(g,P)-L(g,P) < \frac{\eps}{2}.
    \]
    To connect these two, note 
    \[
        \inf\ps{f(x)+g(x)} \geq \inf\ps{f(x)} + \inf\ps{g(x)},\quad (\forall x\in S)
    \]
    \[
        m(f+g,S) \geq m(f,S) + m(g,S),
    \]
    \[
        L(f+g,P) \geq L(f,P)+ L(g,P), \qquad U(f+g,P) \leq U(f,P)+U(g,P).
    \]
    Therefore 
    \[
        U(f+g,P) - L(f+g,P) < \eps,
    \]
    as desired.
    To show the specific value, we use the same partition $P$.
    We find that 
    \[
        U(f,P) + U(g,P) < L(f,P) + L(g,P) + \eps
    \]
    by the last equation. Thus
    \begin{align*}
        \int_a^b f+g &= U(f+g) \leq U(f+g,P) \\
        &\leq U(f,P)+U(g,P) < L(f,P) + L(g,P) + \eps \\
        &\leq L(f) + L(g) + \eps \\
        &= \pp{\int_a^b f + \int_a^b g} +\eps.
    \end{align*}
    Similarly, 
    \[
        \int_a^b f+g > \pp{\int_a^b f + \int_a^b g} - \eps.
    \]
    Since $\eps$ is arbitrary, $\int_a^b f+g = \int_a^b f+\int_a^b g$.
\end{proof}

\begin{proposition}
    \label{prop:integralfunctioninequality}
    If $f$ and $g$ are integrable on $[a,b]$ and $f(x)\leq g(x) \forall x\in[a,b]$, then $\int_a^bf\leq\int_a^bg$.
\end{proposition}

\begin{proof}
    \cref{prop:riemannintegrationprop} shows that $h=g-f$ is integrable on 
    $[a,b]$. $h(x)\geq 0 \forall x\in[a,b]$ implies $L(h,P)\geq 0$ for any 
    partition $P$. So
    \[
        \int_a^bg - \int_a^b f = \int_a^b h = L(h) \geq 0,
    \]
    from which the inequality follows.
\end{proof}

\begin{proposition}
    [``Triangle'' inequality for integrals]
    If $f$ is integrable on $[a,b]$, then $\pa{f}$ is integrable on $[a,b]$
    with $\pa{\int_a^b f}\leq \int_a^b\pa{f}$.
\end{proposition}

\begin{proof}
    Consider $S \subseteq [a,b]$. 
    \begin{align*}
        M(\pa{f},S)-m(\pa{f},S) &= \sup\ps{\pa{f(x)} : x\in S} - \inf\ps{\pa{f} : x\in S} \\
        &= \sup\ps{\pa{f(x)} : x\in S} + \sup\ps{-\pa{f} : x\in S} \\ 
        &= \sup\ps{\pa{f(x)} - \pa{f(y)} : x,y\in S} \\
        &\leq \sup\ps{\pa{f(x) - f(y)} : x,y\in S} \\
        &= \sup\ps{f(x) - f(y):x,y\in S} \\
        &= M(f,S) - m(f,S).
    \end{align*}
    Thus 
    \[
        U(|f|,P) - L(|f|,P) \leq U(f,P) - L(f,P).
    \]
    Choose $\eps>0$. Then $\exists P$ s.t. 
    \[
        U(|f|,P) - L(|f|,P) \leq U(f,P) - L(f,P) < \eps.
    \]

    Since $-|f| \leq f \leq |f|$, it follows from \cref{prop:integralfunctioninequality} that 
    \[
        -\int_a^b|f| \leq \int_a^bf \leq \int_a^b|f| \implies 
        \pa{\int_a^b f}\leq \int_a^b\pa{f}\qedhere
    \]
\end{proof}

\begin{proposition}
    Let $f$ be on $[a,b]$ If $a<c<b$ and $f$ is integrable on $[a,c]$ 
    and $[c,b]$, then $f$ is integrable on $[a,b]$ and 
    $\int_a^bf = \int_a^cf+\int_c^bf$.
\end{proposition}

\begin{definition}
    A function is \vocab{piecewise monotonic} if $\exists$ a partition 
    $P = \ps{a=t_0<\cdots<t_n=b}$ s.t. $f$ is monotonic on $(t_{k-1}, t_k)$ 
    for $1\leq k\leq n$. 
    
    The function is \vocab{piecewise continuous}
    if $\exists$ a partition $P$ of $[a,b]$ s.t. $f$ is 
    uniformly continuous on $(t_{k-1},t_k)$.
\end{definition}

Both of these types of functions are integrable.

\clearpage

\section{December 6th, 2022}

\subsection{Intermediate value theorem for integrals}

\begin{theorem}
    [IVT for integrals]
    If $f$ is a continuous function on $[a,b]$, then 
    for at least one $x\in[a,b]$,
    \[
        f(x) = \frac{1}{b-a}\int_a^bf.
    \]
\end{theorem}
$f(x)$ is the averange value of the function on $[a,b]$.

\begin{proof}
    Let $m$ and $M$ be the minimum and maximum of $f$ on $[a,b]$ respectively.
    If $m=M$, then $f$ is constant and the result holds for all $x\in[a,b]$.
    
    Otherwise $m<M$, and then $\exists x_0\neq y_0$ s.t. 
    $f(x_0) = m, f(y_0) = M$. Consider $M-f$ and $f-m$, which are non-negative and not identically zero.
    By previous results, $\int_{a}^{b}M-f\geq 0$, $\int_{a}^{b}f-m\geq 0$.
    Moreover the inequality strict, since $f$ is continuous (see HW8 Q1).
    Therefore 
    \begin{align*}
        \int_{a}^{b}m < &\int_{a}^{b}f < \int_{a}^{b}M \\
        (b-a)m < &\int_{a}^{b}f<(b-a)M\\
        m < &\frac{1}{b-a}\int_{a}^{b}f<M.
    \end{align*}
    Apply \cref{thm:intermediatevalthm} between $x_0, y_0$ to get 
    the desired $x$.
\end{proof}

\subsection{Fundamental theorems of calculus}

\begin{theorem}
[The fundamental theorem of calculus]
    \label{thm:fundthmcalc}
    If $g$ is continuous on $[a,b]$ and differentiable on $(a,b)$
    and $g'$ integrable on $[a,b]$, then 
    \[
        \int_{a}^{b}g' = g(b)-g(a).
    \]
\end{theorem}

\begin{proof}
    Choose $\eps>0$. $\exists P=\ps{a=t_0<\cdots<t_n=b}$ of the interval 
    $[a,b]$ s.t. 
    \[
        U(g',P) - L(g',P) < \eps.
    \]
    Apply \cref{thm:mvt} to each interval $[t_{k-1},t_k]$.
    $\exists x_k\in (t_{k-1}, t_k)$ s.t. 
    \[
        (t_k-t_{k-1})g'(x_k) = g(t_k) - g(t_{k-1}).
    \]
    Hence 
    \begin{align*}
        g(b)-g(a) &= \sum_{k=1}^{n}(g(t_k) - g(t_{k-1})) \\
        &= \sum_{k=1}^{n}g'(x_k)(t_k-t_{k-1}),
    \end{align*}
    giving us 
    \[
        \sum_{k=1}^{n}m(g',[t_{k-1}, t_k])(t_k-t_{k-1}) \leq g(b)-g(a) \leq \sum_{k=1}^{n}M(g',[t_{k-1}, t_k])(t_k-t_{k-1}).
    \]
    So
    \begin{align*}
        L(g',P) \leq g(b)-g(a) &\leq  U(g',P), \\
        L(g',P) \leq \int_{a}^{b}g' &\leq  U(g',P),
    \end{align*}
    are both true. 
    Therefore 
    \[
        \pa{\int_{a}^{b}g' - (g(b)-g(a))} < \eps.
    \]
    Since $\eps$ is arbitrary, $\int_{a}^{b}g'=g(b)-g(a)$.
\end{proof}

\begin{theorem}
[Integration by parts]
    \label{thm:intbyparts}
    Suppose that $u,v$ continuous on $[a,b]$ and differentiable on $(a,b)$.
    If $u', v'$ integrable on $[a,b]$, then 
    \[
        \int_{a}^{b}uv' - \int_{a}^{b}u'v = 
        u(b)v(b)-u(a)v(a).
    \]
\end{theorem}

\begin{proof}
    Let $g=uv \implies g'=uv'+u'v$. Products on integrable functions 
    are integrable (exercise), so 
    \[
        \int_{a}^{b}g' = g(b)-g(a) = u(b)v(b) - u(a)v(a).
    \]
    Replacing $g'$ with $uv'+u'v$ finishes.
\end{proof}

\begin{theorem}
    [The fundamental theorem of calculus II]
    \label{thm:fundthmcalc2}
    Let $f$ be integrable on $[a,b]$. For $x\in[a,b]$, let 
    $F(x) = \int_{a}^{x}f(t)dt$. Then 
    \begin{enumerate}
        \item $F$ is continuous on $[a,b]$.
        \item If $f$ continuous at $x_0\in(a,b)$, then $F$ is differentiable 
        at $x_0$ and $F'(x_0) = f(x_0)$.
    \end{enumerate}
\end{theorem}

\begin{proof}
    Choose $B>0$ s.t. $\pa{f(x)}\leq B$. If $x,y\in [a,b]$ $x<y$ with 
    $\pa{x-y} < \frac{\eps}{B}$, then 
    \[
        \pa{F(y)-F(x)} = \pa{\int_{x}^{y}f(t)dt} \leq \int_{a}^{y}\pa{f(t)}dt \leq \int_{x}^{y}Bdt = B(y-x) < \eps.
    \]
    $\implies F$ is \textit{uniformly} continuous.

    Suppose $f$ continuous at $x_0\in(a,b)$. Then 
    \[
        \frac{F(x)-F(x_0)}{x-x_0} = \frac{1}{x-x_0}\int_{x_0}^{x}f(t)dt,
    \]
    and
    \[
        f(x_0) = \frac{1}{x-x_0}\int_{x_0}^{x}f(x_0)dt.
    \]
    \begin{align*}
        \frac{F(x) - F(x_0)}{x-x_0} - f(x_0) = \frac{1}{x-x_0}\int_{x_0}^{x}f(t)-f(x_0)dt.
    \end{align*}
    Choose $\eps>0$. Since $f$ continuous, $\exists \delta>0$ s.t. 
    $t\in(a,b)$ and $\pa{t-x_0} < \delta$, then $\pa{f(t)-f(x_0)}<\eps$.
    Therefore 
    \[
        \pa{\frac{F(x)-F(x_0)}{x-x_0}-f(x_0)} \leq \eps 
        \implies 
        F'(x_0) = \lim_{x\rightarrow x_0}\frac{F(x)-F(x_0)}{x-x_0} = f(x_0).
    \]
    It seems inconvenient that the first inequality is $\leq \eps$ vs. 
    $<\eps$, but we can see that if we take $\eps\mapsto \frac{\eps}{2}$, then 
    the proof can conclude with a strict ``less than.''
\end{proof}

\subsection{Change of variables}
\begin{theorem}
    [Change of variables]
    \label{thm:changevars}
    Let $u$ be a differentiable function on the open interval $J$ s.t. 
    $u'$ continuous and let $I$ be an open interval s.t. $u(J)= I$ 
    If $f$ is continuous on $I$, then $f \circ u$ is continuous on $J$ 
    and
    \[
        \int_{a}^{b}f\circ u(x) u'(x)dx = \int_{u(a)}^{u(b)}f(u)du 
    \]
    for $a,b\in J$.
\end{theorem}

\begin{proof}
    $f \circ u $ is continuous by previous result. Chooose $c\in I$ and 
    let $F(u) = \int_{c}^{u}f(t)dt$. Then $F'(u)=f(u)$ by 
    \cref{thm:fundthmcalc2}. Let $f=F\circ u$. Then 
    \[
        g'(x) = F'(u(x))u'(x) = f(u(x))u'(x).
    \]
    Then 
    \begin{align*}
        \int_{a}^{b}f\circ u(x)u'(x)dx &= \int_{a}^{b}g'(x)dx  \\
        &= g(b)-g(a) \\
        &= F(u(b))-F(u(a)) \\
        &= \int_{c}^{u(b)}f(t)dt - \int_{c}^{u(a)}f(t)dt = \int_{u(a)}^{u(b)}f(t)dt \qedhere
    \end{align*}
\end{proof}

\subsection{Improper integrals}

% \begin{theorem}
% [Improper integrals]
% \label{thm:impropintegral}
Consider an interval $[a,b)$ where $b\in \RR\cup\ps{\infty}$. 
Suppose $f$ is a function that is integrable on each $[a,d]$ 
for $a<d<b$ and 
\[
    \lim_{d\rightarrow b^-}\int_{a}^{d}f(x)dx 
\]
evaluates to a number in $\ol{\RR}$.
Then 
\[
    \int_{a}^{b}f(x)dx = \lim_{d\rightarrow b^-}\int_{a}^{d}f(x)dx.
\]
If the interval is $(a,b]$ instead and $f$ is integrable on $[c,b]$ for all $a<c<b$, then 
\[
    \int_{a}^{b}f(x)dx = \lim_{c\rightarrow a^+}\int_{c}^{b}f(x)dx.
\]
For $f$ on $(a,b)$ and integrable on all closed subintervals $[c,d]$,
then 
\[
    \int_{a}^{b}f(x)dx = \int_{a}^{\alpha}f(x)dx + \int_{\alpha}^{b}f(x)dx, \qquad \alpha\in(a,b).
\]
\clearpage

\section{December 8th, 2022}

\subsection{Cauchy principal value}

Consider 
\[
    \int_{-\infty}^{\infty}\frac{x}{1+x^2}dx. 
\]
As $x\to \infty$, the function gets close to $\frac{1}{x}$, and 
$x\to-\infty$, the function gets close to $-\frac{1}{x}$.
By the regular definition, we can 

\begin{align*}
    \int_{-\infty}^{\infty}\frac{x}{1+x^2}dx &= \int_{-\infty}^{\alpha}\frac{x}{1+x^2}dx + \int_{\alpha}^{\infty}\frac{x}{1+x^2}dx \\ 
    &= \underbrace{\lim_{a\rightarrow -\infty}\int_{a}^{\alpha}\frac{x}{1+x^2}dx}_{-\infty} +
    \underbrace{\lim_{b\rightarrow \infty}\int_{\alpha}^{b}\frac{x}{1+x^2}dx}_\infty.
\end{align*}

So the integral value does not exist. Using the \vocab{Cauchy principal value}, we take the limit concurrently to $-\infty$ and $\infty$.

\begin{align*}
    P \int_{-\infty}^{\infty}\frac{x}{1+x^2}dx &= \lim_{a\rightarrow \infty}\int_{-a}^{a}\frac{x}{1+x^2}dx\\ 
    &= \lim_{a\rightarrow \infty}0\\
    &= 0
\end{align*}

\subsection{Continuity in metric spaces}

Consider metric spaces $(S,d), (S^*,d^*)$. Consider maps $f : S \rightarrow S^*$.
\begin{definition}
    $f : S \rightarrow S^*$ is \vocab{continuous} at $s_0\in S$ if 
    \[
        \forall \eps>0, \exists \delta>0 \text{ s.t. } d(s,s_0) < \delta \implies d^*(f(s), f(s_0)) < \eps.
    \]
    A function $f$ is \vocab{continuous} on $E \subseteq S$ if $f$ 
    is continuous at each point of $E$.
    A function is \vocab{uniformly continuous} on $E \subseteq S$ if 
    \[
        \forall \eps>0, \exists \delta>0 \text{ s.t. if } s,t\in E 
        \text{ and } d(s,t) < \delta \implies d^*(f(s), f(t)) < \eps.
    \]
\end{definition}

If $S=S^* = \RR$ and $d=d^*$ is the Euclidean metric, then these match 
the typical definitions.

\begin{definition}
    A \vocab{path} is a continuous mapping $\gamma : \RR \rightarrow \RR^k$.
    The image $\gamma(\RR)$ is called a \vocab{curve}.
\end{definition}

\begin{example}
    An ellipse is a path: $\gamma(t) = (a\cos t, b\sin t)$.
\end{example}

\begin{proposition}
    If $f_1,f_2,\dots,f_k$ are continuous functions that are real-valued
    $(\RR\to \RR)$, then 
    \[
        \gamma(t) = (f_1(t),\dots, f_k(t))
    \]
    defines a path in $\RR^k$.
\end{proposition}

\begin{proof}
    We need to show that $\gamma$ is continuous. Pick 
    $\vec{x}, \vec{y}\in \RR^k$. Then 
    \begin{align*}
        d^*(\vec{x},\vec{y}) &= \pp{\sum_{j=1}^{k}(x_j-y_j)^2}^{\frac{1}{2}}\\
        &\leq \pp{k \max_{j=1}^k (x_j-y-j)^2}^{\frac{1}{2}} \\
        &= \sqrt{k} \max_{j=1}^k \pa{x_j-y_j}.
    \end{align*}
    Consider $t_0\in \RR$ and $\eps>0$. For $j=1,\dots,k$, 
    $\exists \delta_j>0$ s.t. 
    \[
        \pa{t-t_0} < \delta_j \implies \pa{f_j(t)-f_j(t_0)} < \frac{\eps}{\sqrt k}.
    \]
    If $\delta = \min\ps{\delta_1,\dots,\delta_k}$, and $\pa{t-t_0} < \delta$,
    then we can find 
    \[
        \ps{\max{\pa{f_j(t)-f_j(t_0)}:1\leq j\leq k}} < \frac{\eps}{\sqrt k},
    \]
    so 
    \[
        d^*(\gamma(t), \gamma(t_0)) \leq \sqrt k  \ps{\max{\pa{f_j(t)-f_j(t_0)}:1\leq j\leq k}} < \eps. \qedhere
    \]
\end{proof}

\begin{theorem}
    Suppose that $(S,d), (S^*, d^*)$ are two metric spaces.
    $f: S\to S^*$ is continuous on $S$ iff 
    $f\inv(U)$ is an open subset of $S$ for every open subset $U$ 
    of $S^*$.
\end{theorem}

\begin{proof}
    Suppose that $f$ is continuous on $S$. Let $U$ be an open 
    subset of $S^*$. Consider any $s_0 \in f\inv(U) \implies 
    f(s_0)\in U$. Since $U$ is open, $\exists \eps>0$ s.t. 
    \[
        \ps{s^*\in S\mid d^*(s^*, f(s_0))<\eps} = N_\eps(f(s_0)) 
        \subseteq U.
    \]
    Since $f$ is continuous at $s_0$, $\exists \delta>0$ s.t. 
    \[
        d(s,s_0) < \delta \implies d^*(f(s),f(s_0)) < \eps \implies f(s) \in U \implies s\in f\inv(U).
    \]
    In other words, $N_\delta(s_0) \subseteq U$. Hence $s_0$ is 
    an interior pt. and $f\inv(U)$ is open.

    Suppose that converse property holds. 
    Consider $s_0\in S$. Choose $\eps>0$, and examine
    $N_\eps(f(s_0))$. Then $F = f\inv(N_\eps(f(s_0)))$ is open 
    by the assumption. For any $s_0\in F$, $\exists \delta>0$ s.t. 
    $N_\delta(s_0) \subseteq F$. Hence if $d(s,s_0) < \delta$, then 
    $d^*(f(s),f(s_0)) < \eps$.
\end{proof}

\begin{theorem}
    Let $(S,d)$ and $(S^*,d^*)$ be metric spaces, and let 
    $f : S \rightarrow S^*$ be continuous. Suppose that $E$ is a 
    compact subset of $S$. Then 
    \begin{enumerate}
        \item $f(E)$ is a compact subset of $S^*$.
        \item f is uniformly continuous on $E$.
    \end{enumerate}
\end{theorem}

\begin{proof}
    (1) Let $\mathcal U$ be an open cover of $f(E)$. For each 
    $U\in \mathcal{U}$, $f\inv(U)$ is open in $S$. Also 
    $\ps{f\inv(U):U\in \mathcal{U}}$ is a cover of $E$. $x\in E \implies f(x)\in f(E)$ and $f(x)\in U'$ for some $U'$ so $x\in f\inv(U')$.

    Since $E$ is compact, $\exists U_1,\dots, U_m \in \mathcal{U}$ s.t. 
    $E \subseteq \bigcup_{i=1}^{m}f\inv(U_i)$. So $\ps{U_1,\dots,U_m}$ is a 
    finite subcover of $f(E)$.

    (2) Choose $\eps>0$. For $s\in E$ $\exists \delta_s>0$ s.t. 
    $d(s,t)< \delta_s \implies d^*(f(s),f(t)) < \frac{\eps}{2}$.
    Define sets $V_s := \ps{t\in S \mid d(s,t)<\frac{\delta_s}{2}}$.
    $\mathcal{V}:=\ps{V_s\mid s\in E}$ is an open cover of $E$. 
    By compactness, $\exists V_{s_1},\dots V_{s_n}$ that covers $E$.

    Define $\delta = \frac{1}{2}\min\ps{\delta_1,\dots,\delta_n}$

    Consider $s,t\in E$ with $d(s,t)<\delta$. Since $s\in V_{s_k}$ for 
    some $s_k$, then $d(s,s_k)<\frac{\delta_{s_k}}{2}$. Then 
    \[
        d(t, s_k) \leq d(t,s) + d(s,s_k) < \delta+\frac{\delta_{s_k}}{2} < \delta_{s_k}.
    \]
    Thus 
    \[
        d(t, s_k), d(s,s_k) < \delta_{s_k} \implies 
        d^*(f(s),f(s_k)), d^*(f(t),f(s_k)) < \frac{\eps}{2},
    \]
    implying 
    \[
        d^*(f(s),f(t)) < \eps.
    \]
\end{proof}

\begin{corollary}
    Let $f : (S,d) \rightarrow \RR$ be continuous and $S$ compact. 
    Then for $E \subseteq S$, 
    \begin{enumerate}
        \item $f$ is bounded on $E$.
        \item $f$ assumes its minimum/maximum on $E$.
    \end{enumerate}
\end{corollary}

\begin{proof}
    $f(E)$ compact on $\RR$ $\implies$ $f(E)$ is closed and bounded 
    by \cref{thm:heineborelthm}.
\end{proof}

\clearpage
\appendix
\section{Appendix}
\subsection{p-norms}
Let the \vocab{$p$-norm} of a vector $z$ be 
\[\pn{z}_p = \sqrt[p]{\sum_{i=1}^{k}\pa{z_i}^p}. \]
Note that the Euclidean norm is just the $2$-norm.

When $p$ is large, \[ \pn{z}_\infty = \max_{i=1}^k 
\ps{\pa{z_i}}. \]
Note now that the solutions to $\pn{\vec{x}}_p=1$ for $p=1$ is a diamond, 
$p=2$ a circle, $p=4$ a superellipse, and $p=\infty$ a square. 

\subsection{History of power series approximations}
To approximate $\pi$, 
James Gregory (1638-1675) created the power series 
\[\tan\inv x = x-\frac{x^3}{3}+\frac{x^5}{5}-\cdots\]
which evalutates
\[\tan\inv 1 =\frac{\pi}{4}= 1-\frac{1}{3}+\frac{1}{5}-\cdots \] 
However, this converges non-exponentially, which may be slow. But 
\[\tan\inv \half = 1-\frac{\pp{\half}^3}{3}+\frac{\pp{\half}^5}{5}-\cdots \qquad \tan\inv \frac{1}{3} = 1-\frac{\pp{\frac{1}{3}}^3}{3}+\frac{\pp{\frac{1}{3}}^5}{5}-\cdots \]
converges faster. Noting that 
\[\frac{\pi}{4}= \tan\inv \half + \tan\inv \frac{1}{3}, \]
we can calculate $\pi$ much faster (this can be proved by complex 
nubmers).
The \vocab{Machin formula} was found in the 18th century: 
\[ \frac{\pi}{4}=4\tan\inv\frac{1}{5}-\tan\inv\frac{1}{239},\]
which John Machin used in 1706 to find $100$ digits,
and Zacharias Dase in 1844 found $200$ digits,
and William Shanks in 1876 found $707$ digits, but everything from $528$
onward was incorrect.

\subsection{Limitations of power series}
Taylor's theorem says 
\[
    f(x)= \sum_{k=0}^{\infty}\frac{\pp{f^{(k)}(x_0)}(x-x_0)^k}{k!},
\]
which allows us to generate power series.
However, it has limitations. Consider 
\[
    f(x) = \begin{cases}
        0 &x\leq 0,\\
        e^{\frac{-1}{x^2}} &x>0.
    \end{cases}
\]
\[
    f^{(n)}(x) = \begin{cases}
        0 &x\leq 0,\\
        p(\frac{1}{x})e^{\frac{-1}{x^2}} &x>0. \\
    \end{cases}
\]
For some polynomial $p(x)$.
The function is infinitely differentiable, but $f^{(n)}(0)=0$ for all $n$,
so a Taylor polynomial won't give a proper approximation.

\subsection{Analysis bingo}
W
\begin{figure}[h]
    \centering
    \includegraphics[width=\textwidth]{images/rycroftbingo.PNG}
\end{figure}
\end{document}